\documentclass[12pt]{article}   %12 point font for Times New Roman


\usepackage{xcolor}
\usepackage{color}
\usepackage{subcaption}
\usepackage{float}
\usepackage{mathrsfs}
\usepackage{revsymb}
\usepackage{booktabs}
\usepackage{multirow}
\usepackage{rotating}
\usepackage{amssymb}
\usepackage{nicefrac}
\usepackage{graphicx}
\usepackage{epsfig}
\usepackage{mathrsfs}
\usepackage{xcolor}
\usepackage{amsmath}
\usepackage{amsthm}
\usepackage{graphicx}  %for images and plots
\usepackage[a4paper, left=2.5cm, right=2.5cm, top=2cm, bottom=2cm]{geometry}
\usepackage{setspace}  %use this package to set linespacing as desired
\usepackage{times}  %set Times New Roman as the font
\usepackage[explicit]{titlesec}  %title control and formatting

\usepackage[titles]{tocloft}  %table of contents control and formatting
\usepackage[backend=bibtex, sorting=none, bibstyle=ieee]{biblatex}  %reference manager
\usepackage[bookmarks=true, hidelinks]{hyperref}
\usepackage[page]{appendix}  %for appendices
\usepackage{rotating}  %for rotated, landscape images
\usepackage[normalem]{ulem}  %for italicized text
\usepackage{tabto}

%\newcommand{\sectionbreak}{\clearpage}

\usepackage[utf8]{inputenc}

\usepackage{amsmath}
\usepackage{verbatim}
\usepackage{hyperref}
\usepackage{url}
\usepackage{pdfcomment}
\usepackage{multirow}
\usepackage{gensymb}
\usepackage{array}

\usepackage[export]{adjustbox}

\usepackage[toc]{glossaries}

\makeglossaries

\usepackage{afterpage}

%\newcommand\blankpage{%
%	\null
%	\thispagestyle{empty}%
%	\addtocounter{page}{-1}%
%	\newpage}

%TABLE STUFF
%END TABLE STUFF

\DeclareUnicodeCharacter{2212}{-}

\DeclareUnicodeCharacter{223C}{~}

\makeatletter
\newcommand\footnoteref[1]{\protected@xdef\@thefnmark{\ref{#1}}\@footnotemark}
\makeatother

\makeatletter
\newcounter{subsubparagraph}[subparagraph]
\renewcommand\thesubsubparagraph{%
	\thesubparagraph.\@arabic\c@subsubparagraph}
\newcommand\subsubparagraph{%
	\@startsection{subsubparagraph}    % counter
	{6}                              % level
	{\parindent}                     % indent
	{3.25ex \@plus 1ex \@minus .2ex} % beforeskip
	{-1em}                           % afterskip
	{\normalfont\normalsize\bfseries}}
\newcommand\l@subsubparagraph{\@dottedtocline{6}{10em}{5em}}
\newcommand{\subsubparagraphmark}[1]{.}
\makeatother

\usepackage[english]{babel}




\makeatletter
\providecommand\add@text{}
\newcommand\tagaddtext[1]{%
  \gdef\add@text{#1\gdef\add@text{}}}% 
\renewcommand\tagform@[1]{%
  \maketag@@@{\llap{\add@text\quad}(\ignorespaces#1\unskip\@@italiccorr)}%
}
\makeatother



\newtheorem{theorem}{Addendum}
%\newlength{\overwritelength}
%\newlength{\minimumoverwritelength}
%\setlength{\minimumoverwritelength}{1cm}
%\newcommand{\overwrite}[3][red]{%
%	\settowidth{\overwritelength}{$#2$}%
%	\ifdim\overwritelength<\minimumoverwritelength%
%	\setlength{\overwritelength}{\minimumoverwritelength}\fi%
%	\stackrel
%	{%
%		\begin{minipage}{\overwritelength}%
%			\color{#1}\centering\small #3\\%
%			\rule{1pt}{9pt}%
%	\end{minipage}}
%	{\colorbox{#1!50}{\color{black}$\displaystyle#2$}}}

\newlength{\overwritelength}
\newlength{\minimumoverwritelength}
\setlength{\minimumoverwritelength}{1cm}
\newcommand{\overwrite}[3][red]{%
	\settowidth{\overwritelength}{$#2$}%
	\ifdim\overwritelength<\minimumoverwritelength%
	\setlength{\overwritelength}{\minimumoverwritelength}\fi%
	\stackrel
	{%
		\begin{minipage}{\overwritelength}%
			\color{#1}\centering\small #3\\%
			\rule{1pt}{9pt}%
	\end{minipage}}
	{\colorbox{#1!50}{\color{black}$\displaystyle#2$}}}

%%%%%%%%%%%%%%%%%%%%%%%%%%%%%%%%%%%
% Bibliography
%%%%%%%%%%%%%%%%%%%%%%%%%%%%%%%%%%%

%Add your bibliography file here
\bibliography{refc1}


% prevent certain fields in references from printing in bibliography
\AtEveryBibitem{\clearfield{issn}}
\AtEveryBibitem{\clearlist{issn}}

\AtEveryBibitem{\clearfield{language}}
\AtEveryBibitem{\clearlist{language}}

\AtEveryBibitem{\clearfield{doi}}
\AtEveryBibitem{\clearlist{doi}}

\AtEveryBibitem{\clearfield{url}}
\AtEveryBibitem{\clearlist{url}}

\AtEveryBibitem{%
	\ifentrytype{online}
	{}
	{\clearfield{urlyear}\clearfield{urlmonth}\clearfield{urlday}}}

%%%%%%%%%%%%%%%%%%%%%%%%%%%%%%%%%%
%paragraph indent
%%%%%%%%%%%%%%%%%%%%%%%%%%%%%%%%%%

\setlength{\parindent}{4em}
\setlength{\parskip}{1em}

%%%
%The subsubsub...subsection format
%%%


\title{Studiul descărcării luminescente. Obținerea curbei lui Paschen}
\author{Ștefan-Răzvan~Anton\\ Anul 3, Grupa 1334,\\ Facultatea de Științe Aplicate}

\begin{document}

\maketitle



{\section{Scopul lucrării}
%
1.  Înțelegerea modului de funcționare a unui tub de descărcare și a parametrilor care îl caracterizează.\\
2.  Aplicarea unei proceduri de măsurare pe un sistem controlat de la distanță.\\
3.  Cooperarea în organizare astfel încat să se atinga obiectivele propuse.\\
4. Trasarea și interpretarea curbei Paschen obținute.
}
%
{\section{Principiul Fizic}
În figura \ref{Montaj} se poate vizualiza diagrama schematică a unui tub de descarcăre electrică în aer la presiune scazută.
Un tub de descărcare electrică constă într-un tub de sticlă sigilat ("Discharge tube") astfel încat să nu existe schimb de gaz cu exteriorul, presiunea dinăuntrul tubului poate fi între 1mTorr și 1kTorr\cite{griffiths2005introduction} și este controlată de o pompă cu vid ("To vacuum pump"). În interiorul tubului se vor afla doi electrozi metalici, între care se aplică o tensiune continuă ce poate fi variată ("High voltage generator"). Această diferență de potențial între cei doi electrozi produce apariția unui camp electric orientat de la anodul pozitiv ("Anode") la catodul negativ ("Perforated cathode"). Câmpul astfel creat accelereaza electronii liberi către anod și ionii pozitivi către catod. Datorita ciocnirii inelatice cu particulele neutre din gaz, dacă electronii accelerati ajung la o energie cinetică suficient de mare acestia pot excita și ioniza particulele neutre.

\begin{figure}[H]
\centering
\includegraphics[width=.85\linewidth]{png2pdf}  
    \caption{Diagrama schematică a unui tub de descarcare electrică în aer la presiune scăzută\cite{hammadi2016employment}.}
    \label{Montaj}
\end{figure}


În continuare vom analiza distribuția radiației luminoase emise în funcție de poziția din tub. Privind în figura \ref{Montaj1} de la stânga la dreapta se observă:

Strălucirea catodică ("cathodic glow"), spațiul în care electronii au căpătat destulă energie pentru a excita atomii din gaz, care se dezexcită cu emisie de lumină.

Spatțiu întunecat catodic ("Cathode dark space"), spațiul în care electronii și mai multă energie, iar acestia nu mai excită atomii ci îi ionizează. Proces în urma caruia rezultă ioni și electroni, dar nu și radiație luminoasă.

Strălucirea negativă ("Negative glow"), spațiul în care electronii se recombină cu ionii pozitivi, proces în urma căruia rezultă radiație luminoasă.

Spațiu negativ Faraday ("Faraday dark space"), spațiul dintre strălucirea negativă și strălucirea pozitivă în care nu se emite radiație luminoasă.

Coloana pozitivă ("Positive column"), spațiul în care numărul ioniilor pozitivii scade, deci electronii vor avea o energie destul de mare pentru a începe din nou să excite atomii care la dezexcitare produc radiație luminoasă.

În cadrul coloanei pozitive se observă striații cauzate de faptul că atomii pot absorbi energie numai în cantitati discrete.





\begin{figure}[H]
\centering
\includegraphics[width=.85\linewidth]{dim}  
    \caption{Distribuția radiației luminoase în tubul de descarcare.}
    \label{Montaj1}
\end{figure}

}




{\section{Montajul Experimental}
Montajul experimental consta intr-un simulator ce permite ajustarea parametriilor tubului de descarcare dupa cum urmeaza:

Prin actionarea butonului din dreapta sus denumit "Lights" se poate porni/opri lumina in incinta.

Prin actionarea glisorului de sus denumit "Electrode Voltage" se poate modifica tensiunea dintre electrozi in intervalul $0-2000[V]$.

Prin actionarea glisorului din stanga jos denumit "Electromagnet" se poate modifica intensitatea campului magnetic creat de cei doi electomagneti in intervalul $0-200$[G].

Prin actionarea glisorului din dreapta jos denumit "Pressure" se poate modifica presiunea din incina in intervalul $20-1000$[mTorr].


\begin{figure}[H]
\centering
\includegraphics[width=.85\linewidth]{simulator}  
    \caption{Fereastra de control a simulatorului.}
    \label{Montaj}
\end{figure}
}

{\section{Modul de lucru}
Se lucrează la urmatoarele distanțe între electrozi: $0.2m$ (grupa 1), $0.35m$ (grupa 2), $0.5m$ (grupa 3) si $0.728cm$ (grupa 4).

Pasul 1: Se calculează valorile presiunii ce va fi fixată în simulator dupa formula
\begin{eqnarray}\label{eqq}
p=\frac{4+n+10(k*n-1)}{d} 
\tagaddtext{[mTorr]}
\end{eqnarray}
unde n este numărul grupei de lucru ($n=\overline{1,4} $), k este numărul măsurătorii ($k=\overline{1,5} $) și d este distanța dintre electrozi aferentă grupei de lucru ($0.2, 0.35, 0.5, 0.728$)[m].

Pasul 2: Se alege prima grupa ($n=1$) și prima măsuratoare ($k=1$) și se introduc valorile aferente în formula \ref{eqq}. Valoarea presiunii obținute se introduce în simulator (se poate alege o valoare de $\pm20\%$ față de valoare calculată), simulatorul va modifica automat distanța dintre electrozi.

Pasul 3: Se măreste tensiunea dintre electozi păna cand se detectează plasma (se ajunge la tensiunea de aprindere, $U_b$). Se notează, într-un tabel, distanța dintre electrozi, presiunea în tub și tensiunea dintre electorzi la momentul detecției.

Pasul 4: Se micsorează tensiunea dintre electrozi pană cand plasma nu mai este vizibilă în tub (se ajunge la tensiunea de stingere, $U_d$). Se notează, într-o altă coloană a tabelului de la pasul 3, tensiunea dintre electrozi la momentul stingerii plasmei.

Se repetă pașii anteriori pentru restul de patru măsuratori și restul de trei grupe. Pentru fiecare grupă se va realiza un tabel separat.

Pasul 5: Se reprezintă grafic dependența tensiuniilor $U_b$ și $U_d$ în funcție de produsul presiunii și distanței dintre electrozi. Pentru fiecare grupă se va realiza un grafic separat.

}


{\section{Rezultate}

În urma parcuregerii pașilor descriși în secțiunea anterioară au rezultat tabelele \ref{tab1}, \ref{tab2}, \ref{tab3}, \ref{tab4} și figuriile \ref{FIGURE LABEL}.

Rezultatele obținute sunt nesatisfacatoare, în cazul grupelor 1 și 3, $U_b(p*d)$ descrie o curbă Paschen, dar în cazul grupelor 2 și 4 valoriile obtinuțe dupa minim cresc mult prea brusc pentru a descrie o curba Paschen, lucru evident daca ne uităm la curba de interpolare, mai ales în cazul grupei 4. Valorile experimentale pentru $U_d(p*d)$ sunt mult mai satisfacatoare acestea aflându-se mereu în intervalul (500-600)[V]. Curbele de interpolare ale acestora sunt aproape drepte, fapt ce este în concordanta cu cele asteptate.

\begin{table}[H]
\begin{center}
\begin{tabular}{|c|c|c|c|c|} 
 \hline
 $d[m]$ & $p[mTorr]$ & $p*d[m*mTorr]$ & $U_b[V]$ & $U_d[V]$ \\ 
 \hline\hline
0.77 & 25 & 19.18 & 1193 & 590  \\
 \hline
 0.69 & 73 & 50.44 & 854 & 534  \\
 \hline
 0.57 & 125 & 71.25 & 965 & 516  \\
 \hline
 0.48 & 175 & 83.83 & 1007 & 501  \\
 \hline
 0.39 & 227 & 88.76 & 1043 & 502  \\
 \hline
\end{tabular}
\caption{\label{tab1}Rezultatul măsuratorilor pentru grupa 1 ($20cm$).}
\end{center}
\end{table}



\begin{table}[H]
\begin{center}
\begin{tabular}{|c|c|c|c|c|} 
 \hline
 $d[m]$ & $p[mTorr]$ & $p*d[m*mTorr]$ & $U_b[V]$ & $U_d[V]$ \\ 
 \hline\hline
0.73 & 45 & 32.85 & 1259 & 487  \\
 \hline
 0.63 & 99 & 62.37 & 735 & 521  \\
 \hline
 0.51 & 160 & 81.60 & 803 & 505  \\
 \hline
 0.39 & 216 & 84.24 & 969 & 498  \\
 \hline
 0.30 & 285 & 85.50 & 1037 & 530  \\
 \hline
\end{tabular}
\caption{\label{tab2}Rezultatul măsuratorilor pentru grupa 2 ($35cm$).}
\end{center}
\end{table}


\begin{table}[H]
\begin{center}
\begin{tabular}{|c|c|c|c|c|} 
 \hline
 $d[m]$ & $p[mTorr]$ & $p*d[m*mTorr]$ & $U_b[V]$ & $U_d[V]$ \\ 
 \hline\hline
0.72 & 53 & 38.16 & 1369 & 471  \\
 \hline
 0.57 & 115 & 65.55 & 750 & 512  \\
 \hline
 0.46 & 175 & 85.50 & 966 & 505  \\
 \hline
 0.36 & 232 & 83.52 & 995 & 502  \\
 \hline
 0.29 & 298 & 86.42 & 1064 & 525  \\
 \hline
\end{tabular}
\caption{\label{tab3}Rezultatul măsuratorilor pentru grupa 3 ($50cm$).}
\end{center}
\end{table}


\begin{table}[H]
\begin{center}
\begin{tabular}{|c|c|c|c|c|} 
 \hline
 $d[m]$ & $p[mTorr]$ & $p*d[m*mTorr]$ & $U_b[V]$ & $U_d[V]$ \\ 
 \hline\hline
0.72 & 52 & 37.44 & 1157 & 517  \\
 \hline
 0.63 & 105 & 66.15 & 696 & 509  \\
 \hline
 0.51 & 161 & 82.11 & 765 & 506  \\
 \hline
 0.40 & 215 & 86.00 & 995 & 485  \\
 \hline
 0.31 & 279 & 86.49 & 1091 & 495  \\
 \hline
\end{tabular}
\caption{\label{tab4}Rezultatul măsuratorilor pentru grupa 4 ($72.8cm$).}
\end{center}
\end{table}


\begin{figure}[H]
\centering
\begin{subfigure}{.45\textwidth}
    \centering
    \includegraphics[width=.95\linewidth]{gr1}  
    \caption{Grupa 1 ($20cm$)}
    \label{SUBFIGURE LABEL 1}
\end{subfigure}
\begin{subfigure}{.45\textwidth}
    \centering
    \includegraphics[width=.95\linewidth]{gr2}  
    \caption{Grupa 2 ($35cm$)}
    \label{SUBFIGURE LABEL 2}
\end{subfigure}
\begin{subfigure}{.45\textwidth}
    \centering
    \includegraphics[width=.95\linewidth]{gr3}  
    \caption{Grupa 3 ($50cm$)}
    \label{SUBFIGURE LABEL 3}
\end{subfigure}
\begin{subfigure}{.45\textwidth}
    \centering
    \includegraphics[width=.95\linewidth]{gr4}  
    \caption{Grupa 4 ($72.8cm$)}
    \label{SUBFIGURE LABEL 4}
\end{subfigure}
\caption{Dependența experimentală a tensiuniilor $U_b$ si $U_d$ pentru cele partru grupe.}
\label{FIGURE LABEL}
\end{figure}

Parametrii setați în simulator sunt niste valori medii, în realitate valoare instantanee a distanței dintre electrozi, presiunii și tensiunii dintre electrozi fluctuează în jurul valorii medii. Acest fapt introduce alte erori în rezultatele experimentale pe langa eroriile cauzate de utilizator precum setarea presiunii la o valoare puțin diferită fața de cea calculata cu formula \ref{eqq}.

Pentru a reduce influența acestor erori am mediat rezultatele celor patru grupe în tabelul\ref{tab5} și am refăcut graficele $U_b(p*d)$ și $U_d(p*d)$ pentru noile valori \ref{FIGURE LABEL 2}. În cazul curbei de interpolare pentru $U_b(p*d)$, se observă o mai mare similitudine cu o curba Pachen teoretică, ne mai fiind observată o crestere bruscă după atingerea minimului. Curba de interpolare pentru $U_d(p*d)$ este aproximată tot printr-o dreaptă, dar are varaibilitatea mai mică, iar valoriile experimentale sunt distribuite mult mai apropae de valoarea asteptată de 510V.


\begin{table}[H]
\begin{center}
\begin{tabular}{|c|c|c|c|c|} 
 \hline
 $d[m]$ & $p[mTorr]$ & $p*d[m*mTorr]$ & $U_b[V]$ & $U_d[V]$ \\ 
 \hline\hline
0.73 & 43.74 & 31.91 & 1244 & 516  \\
 \hline
 0.63 & 98.00 & 61.12 & 758 & 519  \\
 \hline
 0.51 & 155 & 78.86 & 874 & 508  \\
 \hline
 0.41 & 209 & 84.39 & 991 & 496  \\
 \hline
 0.32 & 272 & 86.79 & 1058 & 513  \\
 \hline
\end{tabular}
\caption{\label{tab5}Rezultatul pentru media celor patru grupelor.}
\end{center}
\end{table}

\begin{figure}[H]
\centering
\includegraphics[width=.85\linewidth]{medie}  
    \caption{Dependența experimentală a tensiuniilor $U_b$ si $U_d$ pentru cele media celor partrua grupe.}
    \label{FIGURE LABEL 2}
\end{figure}
}



{\section{Studiul eroriilor}
Am amintit mai sus tipurile de erori ce influențează valorile obținute experimental, dar nu am stabilit în ce măsura acestea afectează rezultatul final. Pentru a obține o cuantificare a erorilor vom măsura tensiunea de aprindere pentru presiunea de $100mTorr$ setata în simulator
\begin{table}[H]
\begin{center}
\begin{tabular}{|c|c|c|} 
 \hline
 $d[m]$ & $p[mTorr]$ &   $U_b[V]$  \\ 
 \hline\hline
0.62 & 101 & 892   \\
 \hline
 0.63 & 101 & 916   \\
 \hline
 0.63 & 99 & 935   \\
 \hline
 0.62 & 101 & 936   \\
 \hline
 0.65 & 99 & 908   \\
 \hline
\end{tabular}
\caption{\label{tab5}Rezultatul măsuratorilor pentru $p=100mTorr$.}
\end{center}
\end{table}

În continuare calcuăm media și abaterea standard și trecem rezultatele în forma
\begin{eqnarray}\nonumber
d=0.63\pm0.01[m]\,, \\ \nonumber
p=100.2\pm 0.97[mTorr]\,, \\ \nonumber
U_b=917.40\pm16.64[V]\,.
\end{eqnarray}

Analizând rezultatul putem spune că erorile (cauzate de simulator) pentru presiune și distantă dintre electrozi sunt nesemnificative, eroarea pentru tensiunea de aprindere este aproximativ $2\%$. Deci erorile introduse de utilizatorul simularii sunt cauza neconcordanței graficelor experimentale \ref{FIGURE LABEL} cu cele teoretice, iar medierea rezultatelor pe fiecare grupă produce un rezulat mai apropiat de realitate.

}



{\section{Concluzii}
În aceast lucrare am studiat aprinderea unei descărcari electrice iîn aer la presiune scăzută. Am determinat tensiuniile de aprindere și de stingere pentru diverse distanțe dintre electrozi și diferite presiuni. Am reprezentat grafic curba Paschen: dependența $U_b$ de produsul dintre dinstanța dintre electrozi și presiunea în incintă. Masurătoriile au fost facute pentru patru grupe de lucuru fiecare cu o valaore a distanței dintre electrozi dferită ($0.2, 0.35, 0.5, 0.728$)[m]. Rezultatele sunt satisfăcatoare doar pentru două dintre cele patru grupe de lucru utilizate, dar medierea valorilor experimentale permite obținerea unei curbe Paschen determinată experimental apropiată de cea teoretică.
}

\printbibliography

\end{document}