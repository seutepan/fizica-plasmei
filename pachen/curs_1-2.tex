\documentclass[12pt]{article}   %12 point font for Times New Roman


\usepackage{xcolor}
\usepackage{color}
\usepackage{subcaption}
\usepackage{float}
\usepackage{mathrsfs}
\usepackage{revsymb}
\usepackage{booktabs}
\usepackage{multirow}
\usepackage{rotating}
\usepackage{amssymb}
\usepackage{nicefrac}
\usepackage{graphicx}
\usepackage{epsfig}
\usepackage{mathrsfs}
\usepackage{xcolor}
\usepackage{amsmath}
\usepackage{amsthm}
\usepackage{graphicx}  %for images and plots
\usepackage[a4paper, left=2.5cm, right=2.5cm, top=2cm, bottom=2cm]{geometry}
\usepackage{setspace}  %use this package to set linespacing as desired
\usepackage{times}  %set Times New Roman as the font
\usepackage[explicit]{titlesec}  %title control and formatting

\usepackage[titles]{tocloft}  %table of contents control and formatting
\usepackage[backend=bibtex, sorting=none, bibstyle=ieee]{biblatex}  %reference manager
\usepackage[bookmarks=true, hidelinks]{hyperref}
\usepackage[page]{appendix}  %for appendices
\usepackage{rotating}  %for rotated, landscape images
\usepackage[normalem]{ulem}  %for italicized text
\usepackage{tabto}

%\newcommand{\sectionbreak}{\clearpage}

\usepackage[utf8]{inputenc}

\usepackage{amsmath}
\usepackage{verbatim}
\usepackage{hyperref}
\usepackage{url}
\usepackage{pdfcomment}
\usepackage{multirow}
\usepackage{gensymb}
\usepackage{array}

\usepackage[export]{adjustbox}

\usepackage[toc]{glossaries}

\makeglossaries

\usepackage{afterpage}

%\newcommand\blankpage{%
%	\null
%	\thispagestyle{empty}%
%	\addtocounter{page}{-1}%
%	\newpage}


%TABLE STUFF

%END TABLE STUFF

\DeclareUnicodeCharacter{2212}{-}

\DeclareUnicodeCharacter{223C}{~}

\makeatletter
\newcommand\footnoteref[1]{\protected@xdef\@thefnmark{\ref{#1}}\@footnotemark}
\makeatother

\makeatletter
\newcounter{subsubparagraph}[subparagraph]
\renewcommand\thesubsubparagraph{%
	\thesubparagraph.\@arabic\c@subsubparagraph}
\newcommand\subsubparagraph{%
	\@startsection{subsubparagraph}    % counter
	{6}                              % level
	{\parindent}                     % indent
	{3.25ex \@plus 1ex \@minus .2ex} % beforeskip
	{-1em}                           % afterskip
	{\normalfont\normalsize\bfseries}}
\newcommand\l@subsubparagraph{\@dottedtocline{6}{10em}{5em}}
\newcommand{\subsubparagraphmark}[1]{.}
\makeatother

\usepackage[english]{babel}

\newtheorem{theorem}{Addendum}
%\newlength{\overwritelength}
%\newlength{\minimumoverwritelength}
%\setlength{\minimumoverwritelength}{1cm}
%\newcommand{\overwrite}[3][red]{%
%	\settowidth{\overwritelength}{$#2$}%
%	\ifdim\overwritelength<\minimumoverwritelength%
%	\setlength{\overwritelength}{\minimumoverwritelength}\fi%
%	\stackrel
%	{%
%		\begin{minipage}{\overwritelength}%
%			\color{#1}\centering\small #3\\%
%			\rule{1pt}{9pt}%
%	\end{minipage}}
%	{\colorbox{#1!50}{\color{black}$\displaystyle#2$}}}

\newlength{\overwritelength}
\newlength{\minimumoverwritelength}
\setlength{\minimumoverwritelength}{1cm}
\newcommand{\overwrite}[3][red]{%
	\settowidth{\overwritelength}{$#2$}%
	\ifdim\overwritelength<\minimumoverwritelength%
	\setlength{\overwritelength}{\minimumoverwritelength}\fi%
	\stackrel
	{%
		\begin{minipage}{\overwritelength}%
			\color{#1}\centering\small #3\\%
			\rule{1pt}{9pt}%
	\end{minipage}}
	{\colorbox{#1!50}{\color{black}$\displaystyle#2$}}}

%%%%%%%%%%%%%%%%%%%%%%%%%%%%%%%%%%%
% Bibliography
%%%%%%%%%%%%%%%%%%%%%%%%%%%%%%%%%%%

%Add your bibliography file here
\bibliography{refc1}


% prevent certain fields in references from printing in bibliography
\AtEveryBibitem{\clearfield{issn}}
\AtEveryBibitem{\clearlist{issn}}

\AtEveryBibitem{\clearfield{language}}
\AtEveryBibitem{\clearlist{language}}

\AtEveryBibitem{\clearfield{doi}}
\AtEveryBibitem{\clearlist{doi}}

\AtEveryBibitem{\clearfield{url}}
\AtEveryBibitem{\clearlist{url}}

\AtEveryBibitem{%
	\ifentrytype{online}
	{}
	{\clearfield{urlyear}\clearfield{urlmonth}\clearfield{urlday}}}

%%%%%%%%%%%%%%%%%%%%%%%%%%%%%%%%%%
%paragraph indent
%%%%%%%%%%%%%%%%%%%%%%%%%%%%%%%%%%

\setlength{\parindent}{4em}
\setlength{\parskip}{1em}

%%%
%The subsubsub...subsection format
%%%


\title{Studiul descărcării luminescente. Obținerea curbei lui Paschen}
\author{Ștefan-Răzvan~Anton\\ Anul 3, Grupa 1334,\\ Facultatea de Științe Aplicate}

\begin{document}

\maketitle



{\section{Scopul lucrării}
%
1.  Realizarea experimentală a unei analize statistice.\\
2.  Determinarea unor mărimi caracteristice unei distribuții statistice.\\
3.  Asemănări și deosebiri între distribuțiile Poisson și Gauss.
}
%
{\section{Principiul Fizic}

}
{\section{Montajul Experimental}

}
{\section{Modul de lucru}

}


{\section{Rezultate}

\begin{table}[H]
\begin{center}
\begin{tabular}{|c|c|c|c|c|} 
 \hline
 $d[m]$ & $p[mTorr]$ & $p*d[m*mTorr]$ & $U_b[V]$ & $U_d[V]$ \\ 
 \hline\hline
0.77 & 25 & 19.18 & 1193 & 590  \\
 \hline
 0.69 & 73 & 50.44 & 854 & 534  \\
 \hline
 0.57 & 125 & 71.25 & 965 & 516  \\
 \hline
 0.48 & 175 & 83.83 & 1007 & 501  \\
 \hline
 0.39 & 227 & 88.76 & 1043 & 502  \\
 \hline
\end{tabular}
\caption{\label{tab1}Rezultatul măsuratoriilor pentru grupa 1 ($20cm$).}
\end{center}
\end{table}



\begin{table}[H]
\begin{center}
\begin{tabular}{|c|c|c|c|c|} 
 \hline
 $d[m]$ & $p[mTorr]$ & $p*d[m*mTorr]$ & $U_b[V]$ & $U_d[V]$ \\ 
 \hline\hline
0.73 & 45 & 32.85 & 1259 & 487  \\
 \hline
 0.63 & 99 & 62.37 & 735 & 521  \\
 \hline
 0.51 & 160 & 81.60 & 803 & 505  \\
 \hline
 0.39 & 216 & 84.24 & 969 & 498  \\
 \hline
 0.30 & 285 & 85.50 & 1037 & 530  \\
 \hline
\end{tabular}
\caption{\label{tab1}Rezultatul măsuratoriilor pentru grupa 2 ($35cm$).}
\end{center}
\end{table}


\begin{table}[H]
\begin{center}
\begin{tabular}{|c|c|c|c|c|} 
 \hline
 $d[m]$ & $p[mTorr]$ & $p*d[m*mTorr]$ & $U_b[V]$ & $U_d[V]$ \\ 
 \hline\hline
0.72 & 53 & 38.16 & 1369 & 471  \\
 \hline
 0.57 & 115 & 65.55 & 750 & 512  \\
 \hline
 0.46 & 175 & 85.50 & 966 & 505  \\
 \hline
 0.36 & 232 & 83.52 & 995 & 502  \\
 \hline
 0.29 & 298 & 86.42 & 1064 & 525  \\
 \hline
\end{tabular}
\caption{\label{tab1}Rezultatul măsuratoriilor pentru grupa 3 ($50cm$).}
\end{center}
\end{table}


\begin{table}[H]
\begin{center}
\begin{tabular}{|c|c|c|c|c|} 
 \hline
 $d[m]$ & $p[mTorr]$ & $p*d[m*mTorr]$ & $U_b[V]$ & $U_d[V]$ \\ 
 \hline\hline
0.72 & 52 & 37.44 & 1157 & 517  \\
 \hline
 0.63 & 105 & 66.15 & 696 & 509  \\
 \hline
 0.51 & 161 & 82.11 & 765 & 506  \\
 \hline
 0.40 & 215 & 86.00 & 995 & 485  \\
 \hline
 0.31 & 279 & 86.49 & 1091 & 495  \\
 \hline
\end{tabular}
\caption{\label{tab1}Rezultatul măsuratoriilor pentru grupa 4 ($72.8cm$).}
\end{center}
\end{table}


\begin{figure}
\centering
\begin{subfigure}{.45\textwidth}
    \centering
    \includegraphics[width=.95\linewidth]{gr1}  
    \caption{Grupa 1 ($20cm$)}
    \label{SUBFIGURE LABEL 1}
\end{subfigure}
\begin{subfigure}{.45\textwidth}
    \centering
    \includegraphics[width=.95\linewidth]{gr2}  
    \caption{Grupa 2 ($35cm$)}
    \label{SUBFIGURE LABEL 2}
\end{subfigure}
\begin{subfigure}{.45\textwidth}
    \centering
    \includegraphics[width=.95\linewidth]{gr3}  
    \caption{Grupa 3 ($50cm$)}
    \label{SUBFIGURE LABEL 3}
\end{subfigure}
\begin{subfigure}{.45\textwidth}
    \centering
    \includegraphics[width=.95\linewidth]{gr4}  
    \caption{Grupa 4 ($72.8cm$)}
    \label{SUBFIGURE LABEL 4}
\end{subfigure}
\caption{Dependența experimentală a tensiuniilor $U_b$ si $U_d$ pentru cele partru grupe}
\label{FIGURE LABEL}
\end{figure}

\begin{figure}
\includegraphics[width=.85\linewidth]{medie}  
    \caption{}
    \label{FIGURE LABEL 1}
\end{figure}
}







{\section{Concluzii}

}
\end{document}