\documentclass[12pt]{article}   %12 point font for Times New Roman
\usepackage{subfig}
\usepackage{xcolor}
\usepackage{color}
\usepackage{float}
\usepackage{mathrsfs}
\usepackage{revsymb}
\usepackage{booktabs}
\usepackage{multirow}
\usepackage{rotating}
\usepackage{amssymb}
\usepackage{nicefrac}
\usepackage{graphicx}
\usepackage{epsfig}
\usepackage{mathrsfs}
\usepackage{xcolor}
\usepackage{amsmath}
\usepackage{amsthm}
\usepackage{graphicx}  %for images and plots
\usepackage[a4paper, left=2.5cm, right=2.5cm, top=2cm, bottom=2cm]{geometry}
\usepackage{setspace}  %use this package to set linespacing as desired
\usepackage{times}  %set Times New Roman as the font
\usepackage[explicit]{titlesec}  %title control and formatting

\usepackage[titles]{tocloft}  %table of contents control and formatting
\usepackage[backend=bibtex, sorting=none, bibstyle=ieee]{biblatex}  %reference manager
\usepackage[bookmarks=true, hidelinks]{hyperref}
\usepackage[page]{appendix}  %for appendices
\usepackage{rotating}  %for rotated, landscape images
\usepackage[normalem]{ulem}  %for italicized text
\usepackage{tabto}

%\newcommand{\sectionbreak}{\clearpage}

\usepackage[utf8]{inputenc}

\usepackage{amsmath}
\usepackage{verbatim}
\usepackage{hyperref}
\usepackage{url}
\usepackage{pdfcomment}
\usepackage{multirow}
\usepackage{gensymb}
\usepackage{array}

\usepackage[export]{adjustbox}

\usepackage[toc]{glossaries}

\makeglossaries

\usepackage{afterpage}

%\newcommand\blankpage{%
%	\null
%	\thispagestyle{empty}%
%	\addtocounter{page}{-1}%
%	\newpage}


%TABLE STUFF

%END TABLE STUFF

\DeclareUnicodeCharacter{2212}{-}

\DeclareUnicodeCharacter{223C}{~}

\makeatletter
\newcommand\footnoteref[1]{\protected@xdef\@thefnmark{\ref{#1}}\@footnotemark}
\makeatother

\makeatletter
\newcounter{subsubparagraph}[subparagraph]
\renewcommand\thesubsubparagraph{%
	\thesubparagraph.\@arabic\c@subsubparagraph}
\newcommand\subsubparagraph{%
	\@startsection{subsubparagraph}    % counter
	{6}                              % level
	{\parindent}                     % indent
	{3.25ex \@plus 1ex \@minus .2ex} % beforeskip
	{-1em}                           % afterskip
	{\normalfont\normalsize\bfseries}}
\newcommand\l@subsubparagraph{\@dottedtocline{6}{10em}{5em}}
\newcommand{\subsubparagraphmark}[1]{.}
\makeatother

\usepackage[english]{babel}

\newtheorem{theorem}{Addendum}
%\newlength{\overwritelength}
%\newlength{\minimumoverwritelength}
%\setlength{\minimumoverwritelength}{1cm}
%\newcommand{\overwrite}[3][red]{%
%	\settowidth{\overwritelength}{$#2$}%
%	\ifdim\overwritelength<\minimumoverwritelength%
%	\setlength{\overwritelength}{\minimumoverwritelength}\fi%
%	\stackrel
%	{%
%		\begin{minipage}{\overwritelength}%
%			\color{#1}\centering\small #3\\%
%			\rule{1pt}{9pt}%
%	\end{minipage}}
%	{\colorbox{#1!50}{\color{black}$\displaystyle#2$}}}

\newlength{\overwritelength}
\newlength{\minimumoverwritelength}
\setlength{\minimumoverwritelength}{1cm}
\newcommand{\overwrite}[3][red]{%
	\settowidth{\overwritelength}{$#2$}%
	\ifdim\overwritelength<\minimumoverwritelength%
	\setlength{\overwritelength}{\minimumoverwritelength}\fi%
	\stackrel
	{%
		\begin{minipage}{\overwritelength}%
			\color{#1}\centering\small #3\\%
			\rule{1pt}{9pt}%
	\end{minipage}}
	{\colorbox{#1!50}{\color{black}$\displaystyle#2$}}}

%%%%%%%%%%%%%%%%%%%%%%%%%%%%%%%%%%%
% Bibliography
%%%%%%%%%%%%%%%%%%%%%%%%%%%%%%%%%%%

%Add your bibliography file here
\bibliography{refc1}


% prevent certain fields in references from printing in bibliography
\AtEveryBibitem{\clearfield{issn}}
\AtEveryBibitem{\clearlist{issn}}

\AtEveryBibitem{\clearfield{language}}
\AtEveryBibitem{\clearlist{language}}

\AtEveryBibitem{\clearfield{doi}}
\AtEveryBibitem{\clearlist{doi}}

\AtEveryBibitem{\clearfield{url}}
\AtEveryBibitem{\clearlist{url}}

\AtEveryBibitem{%
	\ifentrytype{online}
	{}
	{\clearfield{urlyear}\clearfield{urlmonth}\clearfield{urlday}}}

%%%%%%%%%%%%%%%%%%%%%%%%%%%%%%%%%%
%paragraph indent
%%%%%%%%%%%%%%%%%%%%%%%%%%%%%%%%%%

\setlength{\parindent}{4em}
\setlength{\parskip}{1em}

%%%
%The subsubsub...subsection format
%%%


\title{Analiza spectroscopică a unor lămpi cu descărcare}
\author{Ștefan-Răzvan~Anton\\ Anul 3, Grupa 1334,\\ Facultatea de Științe Aplicate}

\begin{document}

\maketitle



{\section{Scopul lucrării}
%
1.  Analiza mișcării unei particule încărcate funcție de sarcina ei într-un câmp magnetic.\\
2.  Utilizarea unui câmp magnetic pentru controlul unei particule încărcate în aproximația 3D.\\
3.  Evidențierea influenței distribuției după energie asupra traiectoriei de mișcare.\\
4.  Utilizarea montajului experimental pentru realizarea unei proceduri de control a traiectoriei particulelor încărcate.
}
%
{\section{Principiul fizic}
Aici fac anmaliza unei particule in functie de sarcina ei in cp mag
}

{\section{Montajul experimental}

\begin{figure}[H]
\centering
  \includegraphics[width=0.9\textwidth]{montaj.png}
  \caption{\label{figs}Montajul experimental.}
\end{figure}
}


{\section{Modul de lucru}

 
}

{\section{Rezultate}

\begin{table}[H]
\begin{center}
\begin{tabular}{|c|c|c|c|c|c|c|} 
 \hline
 Nr. crt & I [A] & b[mT]  &E[keV] & N(imp) &  n' [Imp/s] & n [imp/s] \\ 
 \hline\hline
1 & 0 & 4.4 & 5.47 &  155 & 2.58 &2.33 \\
 \hline
 2 & 0.1 & 15.4 & 21.56 &  258 & 4.3 & 4.05  \\
 \hline
 3 & 0.2 &  24.5 & 47.34 & 336 & 5.6 & 5.35 \\
 \hline
 4 & 0.3 &34.7 & 81.55 &534& 8.9&8.65 \\
 \hline
 5 & 0.4 & 45.7 & 122.83 &789 &13.15 & 12.9\\
 \hline
 6 & 0.5 & 56.1 & 169.89 &1008 & 16.8 &16.55 \\
 \hline
 7 & 0.6 & 65.8 & 221.62 &1140 & 19 &18.75 \\
 \hline
 8 & 0.7 & 78 & 277.11 & 1267 & 21.11 &20.86 \\
 \hline
 9 & 0.8 & 87 & 335.61 & 1296 & 21.6 &21.35 \\
 \hline
 10 & 0.9 & 97.4 & 396.53 & 1278 & 21.3 &21.05 \\
 \hline
 11 & 1.0 & 107.4 & 459.43 &1076 & 17.93 &17.68 \\
 \hline
 12 & 1.1 & 120.2 & 523.94 &1004 & 16.73 &16.48 \\
 \hline
 13 & 1.2 & 128.5 & 589.79 &894 & 14.9 &12.65 \\
 \hline
 14 & 1.3 & 140 & 656.74 &735 & 12.25 &12 \\
 \hline
 15 & 1.4 & 149 & 724.61 &618 & 10.3 &10.05 \\
 \hline
 16 & 1.5 & 159.3 & 793.27 &477 & 7.95 &7.7 \\
 \hline
 17 & 1.6 & 168.1 & 861.58 &412 & 6.86 &6.61 \\
 \hline
 18 & 1.7 & 174.7 & 932.47 &308 & 5.13 &4.88 \\
 \hline
\end{tabular}
\caption{\label{tab1}Datele colectate pentru sursa de ${}^{90}Sr$ atunci când borna pozitivă a sursei este conectată la borna negativă a sursei.}
\end{center}
\end{table}


\begin{table}[H]
\begin{center}
\begin{tabular}{|c|c|c|c|c|c|c|} 
 \hline
 Nr. crt & I [A] & b[mT]  &E[keV] & N[imp] &  n' [Imp/s] & n [imp/s] \\ 
 \hline\hline
1 & 0 & 4.4 & 5.47 &  82 & 1.36 &1.11 \\
 \hline
 2 & 0.1 & 15.4 & 21.56 &  81 & 1.35 & 1.1  \\
 \hline
 3 & 0.2 &  24.5 & 47.34 & 70 & 1.16 & 0.91 \\
 \hline
 4 & 0.3 &34.7 & 81.55 &57& 0.95&0.7 \\
 \hline
 5 & 0.4 & 45.7 & 122.83 &43 &0.71 & 0.46\\
 \hline
 6 & 0.5 & 56.1 & 169.89 &38 & 0.63 &0.38 \\
 \hline
 7 & 0.6 & 65.8 & 221.62 &31 & 0.51 &0.26 \\
 \hline
 8 & 0.7 & 78 & 277.11 & 34 & 0.56 &0.31 \\
 \hline
 9 & 0.8 & 87 & 335.61 & 34 & 0.56 &0.31 \\
 \hline
 10 & 0.9 & 97.4 & 396.53 & 24 & 0.4 &0.15 \\
 \hline
 11 & 1.0 & 107.4 & 459.43 &30 & 0.5 &0.25 \\
 \hline
 12 & 1.1 & 120.2 & 523.94 &36 & 0.6 &0.35 \\
 \hline
 13 & 1.2 & 128.5 & 589.79 &19 & 0.31 &0.06 \\
 \hline
 14 & 1.3 & 140 & 656.74 &20 & 0.33 &0.08\\
 \hline
 15 & 1.4 & 149 & 724.61 &15 & 0.25 &0 \\
 \hline
 16 & 1.5 & 159.3 & 793.27 &20 & 0.33 &0.08 \\
 \hline
 17 & 1.6 & 168.1 & 861.58 &16 & 0.26 &0.01 \\
 \hline
 18 & 1.7 & 174.7 & 932.47 &15 & 0.25 &0 \\
 \hline
\end{tabular}
\caption{\label{tab1}Datele colectate pentru sursa de ${}^{90}Sr$ atunci când borna pozitivă a sursei este conectată la borna pozitiv㧠a sursei.}
\end{center}
\end{table}



\begin{figure}[H]
\centering
\begin{subfigure}{.45\textwidth}
    \centering
    \includegraphics[width=.95\linewidth]{pm.png}  
    \caption{Grupa 1 ($20cm$)}
    \label{SUBFIGURE LABEL 1}
\end{subfigure}
\begin{subfigure}{.45\textwidth}
    \centering
    \includegraphics[width=.95\linewidth]{pp.png}  
    \caption{Grupa 2 ($35cm$)}
    \label{SUBFIGURE LABEL 2}
\end{subfigure}
\caption{Dependența experimentală a tensiuniilor $U_b$ si $U_d$ pentru cele partru grupe.}
\label{FIGURE LABEL}
\end{figure}

În continuare, utilizând curba de etalonare obținută anterior împreuna cu datele experimentale calculăm lungimea de undă experimentală ($\lambda_e$) pentru lampa cu descărcare în Cadmiu (tabelul \ref{tab2}) și pentru lampa cu descărcare în Zinc (tabelul \ref{tab3}).

\begin{table}[H]
\begin{center}
\begin{tabular}{|c|c|c|c|c|c|c|} 
 \hline
 Culoare & Unghi [grade] & $\lambda_t$[nm]  & $\lambda_e$ [nm] & Tranziție & $\Delta$ E[eV] & $E_{foton}$[eV] \\ 
 \hline\hline
 roșu & 124 & 643 & 620 &  ${5}^{1}D_2 \rightarrow {5}^{1}P_1$&  1.9251&1.9997  \\
 \hline
 verde & 125.6 & 508 & 529 &  ${6}^{3}S_1 \rightarrow {5}^{3}P_2$  & 2.4372 & 2.3437\\
 \hline
 albastru & 126.25 & 479 & 492 &  ${6}^{3}S_1 \rightarrow {5}^{3}P_1$  & 2.5823 &2.5200 \\
 \hline
 ablastru & 126.65 & 467 & 469 &  ${6}^{3}S_1 \rightarrow {5}^{3}P_0$&2.6495  &2.6435  \\
 \hline
 mov & 127.2 & 441 & 438 & ${6}^{1}S_0 \rightarrow {5}^{3}P_1$  &2.8087  &2.8307 \\
 \hline
\end{tabular}
\caption{\label{tab2}Determinarea lungimii de undă pentru lampa cu descărcare în Cadmiu.}
\end{center}
\end{table}


\begin{table}[H]
\begin{center}
\begin{tabular}{|c|c|c|c|c|c|c|} 
 \hline
 Culoare & Unghi [grade] & $\lambda_t$[nm]  & $\lambda_e$ [nm] & Tranziție & $\Delta$ E[eV] & $E_{foton}$[eV] \\ 
 \hline\hline
 roșu & 123.85 & 636 & 629 &  ${4}^{1}D_2 \rightarrow {4}^{1}P_1$ & 1.9482 &1.9711 \\
 \hline
 verde & 125.5 & 518 & 535 &  ${6}^{1}S_0 \rightarrow {4}^{1}P_1$  &2.3919  &2.3174 \\
 \hline
 albastru & 126.05 & 481 & 504 &  ${5}^{3}S_1 \rightarrow {4}^{3}P_2$ &2.5766  & 2.4600 \\
 \hline
 ablastru & 126.55 & 472 & 475 &  ${5}^{3}S_1 \rightarrow {4}^{3}P_1$&2.6248  &  2.6102\\
 \hline
 albastru & 126.8 & 468 & 461 & ${5}^{3}S_1 \rightarrow {4}^{3}P_0$& 2.6484 &  2.6894 \\
 \hline
\end{tabular}
\caption{\label{tab3}Determinarea lungimii de undă pentru lampa cu descărcare în Zinc.}
\end{center}
\end{table}

Pentru identificarea elementelor chimice din plasma de descărcare se pot compara specrele înregistrate experimental(figuriile \ref{heliu} \ref{cadmiu} \ref{zinc}) cu o bază de date a spectrelor a tuturor elementelor chimice.


\begin{figure}[H]
\centering
  \includegraphics[width=0.9\textwidth]{heliu}
  \caption{\label{heliu}Spectrul pentru plasma de descărcare în Heliu.}
\end{figure}

\begin{figure}[H]
\centering
  \includegraphics[width=0.9\textwidth]{cadmiu}
  \caption{\label{cadmiu}Spectrul pentru plasma de descărcare în Cadmiu.}
\end{figure}

\begin{figure}[H]
\centering
  \includegraphics[width=0.9\textwidth]{zinc}
  \caption{\label{zinc}Spectrul pentru plasma de descărcare în Zinc.}
\end{figure}

Structura fină, divizarea liniilor spectrale ale atomilor din cauza spinului electronului și corecțiilor relativiste la ecuația Schrödinger pot fi observate în tabelele \ref{tab2} și \ref{tab3}, pentru unele tranziții se observa structura fină ${6}^{3}S_1 \rightarrow {5}^{3}P_2$, ${6}^{3}S_1 \rightarrow {5}^{3}P_1$ si  ${6}^{3}S_1 \rightarrow {5}^{3}P_0$ pentru Cadmiu și tranzițiile ${5}^{3}S_1 \rightarrow {4}^{3}P_2$, ${5}^{3}S_1 \rightarrow {4}^{3}P_1$ si ${5}^{3}S_1 \rightarrow {4}^{3}P_0$ pentru Zinc. Pentru aceste tranziții, dacă ne uităm la coloana $\Delta$ E ar trebuii să observăm că valoriile sunt foarte apropiate, lucru care se confirmă. De asemenea același comportament se observă la $\Delta$ E  experimental (adica energia fotonului emis $E_{foton}$).

Eroarea pătratica medie a lungimiilor de undă determinate experimental este 15.41 pentru Heliu, 15.17 pentru Cadmiu și 13.60 pentru Zinc. Surprinzător este faptul că eroarea pentru Heliu este mai mare decat pentru Zinc și Cadmiu chiar dacă etalonarea s-a făcut după valoriile experimentale ale Heliului. Un motiv pentru acest fenomen poate fi faptul că la citirea unghiului pentru Heliu nu s-a privit perpendicular pe planul riglei unghiulare. Această eroare de citire de câteva zecimi de grade poate să fi influențat rezultatul. O altă explicație esta că, din grabă, 'ținta' de pe telesopul cu care se localizează liniile spectrale nu a fost poziționată corespunzator, în special pentru liniile care nu au o intensitate puternică.


}






{\section{Concluzii}
În această lucrarea am studiat funcționarea lămpiilor cu descărcare în Heliu, Cadmiu și Zinc. Am determinat experimental unghiul la care apar liniile spectrale ale plasmei de descărcare și prin etalonarea cu ajutorul lampii cu descărcare în Heliu am determinat lungimea de undă a liniilor spectrale a celorlalte două lămpi. Am observat stuctura fină în lampiile cu descărcare în Cadmiu și Zinc. Am propus un procedeu pentru identificarea elementelor chimice din plasma de descărcare.
}
\end{document}