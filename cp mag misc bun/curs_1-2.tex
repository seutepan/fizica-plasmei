\documentclass[12pt]{article}   %12 point font for Times New Roman


\usepackage{xcolor}
\usepackage{color}
\usepackage{subcaption}
\usepackage{float}
\usepackage{mathrsfs}
\usepackage{revsymb}
\usepackage{booktabs}
\usepackage{multirow}
\usepackage{rotating}
\usepackage{amssymb}
\usepackage{nicefrac}
\usepackage{graphicx}
\usepackage{epsfig}
\usepackage{mathrsfs}
\usepackage{xcolor}
\usepackage{amsmath}
\usepackage{amsthm}
\usepackage{graphicx}  %for images and plots
\usepackage[a4paper, left=2.5cm, right=2.5cm, top=2cm, bottom=2cm]{geometry}
\usepackage{setspace}  %use this package to set linespacing as desired
\usepackage{times}  %set Times New Roman as the font
\usepackage[explicit]{titlesec}  %title control and formatting

\usepackage[titles]{tocloft}  %table of contents control and formatting
\usepackage[backend=bibtex, sorting=none, bibstyle=ieee]{biblatex}  %reference manager
\usepackage[bookmarks=true, hidelinks]{hyperref}
\usepackage[page]{appendix}  %for appendices
\usepackage{rotating}  %for rotated, landscape images
\usepackage[normalem]{ulem}  %for italicized text
\usepackage{tabto}

%\newcommand{\sectionbreak}{\clearpage}

\usepackage[utf8]{inputenc}

\usepackage{amsmath}
\usepackage{verbatim}
\usepackage{hyperref}
\usepackage{url}
\usepackage{pdfcomment}
\usepackage{multirow}
\usepackage{gensymb}
\usepackage{array}

\usepackage[export]{adjustbox}

\usepackage[toc]{glossaries}

\makeglossaries

\usepackage{afterpage}

%\newcommand\blankpage{%
%	\null
%	\thispagestyle{empty}%
%	\addtocounter{page}{-1}%
%	\newpage}

%TABLE STUFF
%END TABLE STUFF

\DeclareUnicodeCharacter{2212}{-}

\DeclareUnicodeCharacter{223C}{~}

\makeatletter
\newcommand\footnoteref[1]{\protected@xdef\@thefnmark{\ref{#1}}\@footnotemark}
\makeatother

\makeatletter
\newcounter{subsubparagraph}[subparagraph]
\renewcommand\thesubsubparagraph{%
	\thesubparagraph.\@arabic\c@subsubparagraph}
\newcommand\subsubparagraph{%
	\@startsection{subsubparagraph}    % counter
	{6}                              % level
	{\parindent}                     % indent
	{3.25ex \@plus 1ex \@minus .2ex} % beforeskip
	{-1em}                           % afterskip
	{\normalfont\normalsize\bfseries}}
\newcommand\l@subsubparagraph{\@dottedtocline{6}{10em}{5em}}
\newcommand{\subsubparagraphmark}[1]{.}
\makeatother

\usepackage[english]{babel}




\makeatletter
\providecommand\add@text{}
\newcommand\tagaddtext[1]{%
  \gdef\add@text{#1\gdef\add@text{}}}% 
\renewcommand\tagform@[1]{%
  \maketag@@@{\llap{\add@text\quad}(\ignorespaces#1\unskip\@@italiccorr)}%
}
\makeatother



\newtheorem{theorem}{Addendum}
%\newlength{\overwritelength}
%\newlength{\minimumoverwritelength}
%\setlength{\minimumoverwritelength}{1cm}
%\newcommand{\overwrite}[3][red]{%
%	\settowidth{\overwritelength}{$#2$}%
%	\ifdim\overwritelength<\minimumoverwritelength%
%	\setlength{\overwritelength}{\minimumoverwritelength}\fi%
%	\stackrel
%	{%
%		\begin{minipage}{\overwritelength}%
%			\color{#1}\centering\small #3\\%
%			\rule{1pt}{9pt}%
%	\end{minipage}}
%	{\colorbox{#1!50}{\color{black}$\displaystyle#2$}}}

\newlength{\overwritelength}
\newlength{\minimumoverwritelength}
\setlength{\minimumoverwritelength}{1cm}
\newcommand{\overwrite}[3][red]{%
	\settowidth{\overwritelength}{$#2$}%
	\ifdim\overwritelength<\minimumoverwritelength%
	\setlength{\overwritelength}{\minimumoverwritelength}\fi%
	\stackrel
	{%
		\begin{minipage}{\overwritelength}%
			\color{#1}\centering\small #3\\%
			\rule{1pt}{9pt}%
	\end{minipage}}
	{\colorbox{#1!50}{\color{black}$\displaystyle#2$}}}

%%%%%%%%%%%%%%%%%%%%%%%%%%%%%%%%%%%
% Bibliography
%%%%%%%%%%%%%%%%%%%%%%%%%%%%%%%%%%%

%Add your bibliography file here
\bibliography{refc1}


% prevent certain fields in references from printing in bibliography
\AtEveryBibitem{\clearfield{issn}}
\AtEveryBibitem{\clearlist{issn}}

\AtEveryBibitem{\clearfield{language}}
\AtEveryBibitem{\clearlist{language}}

\AtEveryBibitem{\clearfield{doi}}
\AtEveryBibitem{\clearlist{doi}}

\AtEveryBibitem{\clearfield{url}}
\AtEveryBibitem{\clearlist{url}}

\AtEveryBibitem{%
	\ifentrytype{online}
	{}
	{\clearfield{urlyear}\clearfield{urlmonth}\clearfield{urlday}}}

%%%%%%%%%%%%%%%%%%%%%%%%%%%%%%%%%%
%paragraph indent
%%%%%%%%%%%%%%%%%%%%%%%%%%%%%%%%%%

\setlength{\parindent}{4em}
\setlength{\parskip}{1em}

%%%
%The subsubsub...subsection format
%%%


\title{Mișcarea particulelor încărcate în câmp magnetic}
\author{Ștefan-Răzvan~Anton\\ Anul 3, Grupa 1334,\\ Facultatea de Științe Aplicate}

\begin{document}

\maketitle




{\section{Scopul lucrării}
%
1.  Analiza mișcării unei particule încărcate in funcție de sarcina ei într-un câmp magnetic omogen.\\
2.  Utilizarea unui câmp magnetic pentru controlul unei particule încărcate în aproximația 3D.\\
3.  Evidențierea influenței distribuției după energie asupra traiectoriei de mișcare.\\
4.  Utilizarea montajului experimental pentru realizarea unei proceduri de control a traiectoriei particulelor încărcate.
}
%
{\section{Principiul fizic}
{\subsection{Analiza mișării unei particule încărcate în funcție de sarcina ei intr-un câmp magnetic}
În aproximația 2D, într-un câmp magnetic omogen, o particulă incărcata se va mișca după o orbită circulară fixă. Astfel, forța Lorentz și forța centrifugă sunt in echilibru
\begin{eqnarray}
q v B=\frac{m v^2}{r}\,.
\end{eqnarray}
Deci, o particulă încărcată se va mișca pe traiectoria unui cerc de rază
\begin{eqnarray}
r=\frac{m v}{qB}\,,
\end{eqnarray}
unde $m$, $v$, $q$ sunt masa, viteza respectiv sarcina particulei, iar B este intensitatea câmpului magnetic.

În figura \ref{mis1} avem două cazuri de mișcare a unei particule încărcate în câmp magentic în funcție de sarcina acesteia. Obserăm că, pentru două sarcini egale dar de semn opus, mișcarea particulei descrie același cerc, dar sensul de parcurgere al acestuia este diferit. Sensul se poate afla utilizând regula mâinii drepte.

\begin{figure}[H]
\centering
\begin{subfigure}{.45\textwidth}
    \centering
    \includegraphics[width=.95\linewidth]{qpozitiv}  
    \caption{Sarcină pozitivă.}
    \label{Sarcină pozitivă}
\end{subfigure}
\begin{subfigure}{.45\textwidth}
    \centering
    \includegraphics[width=.95\linewidth]{qnegativ}  
    \caption{Sarcină negativă.}
    \label{Sarcină negativă}
\end{subfigure}
\caption{Mișcarea într-un câmp magnetic omogen, ce iese din foaie, a două particule de sarcină egala și semn contrar.}
\label{mis1}
\end{figure}
}

Observăm și faptul că raza cercului pe care se mișcă particula încărcată este invers proporțională cu sarcina particulei. În figura \ref{mis2} se observă reducerea la jumâtate a razei cercului atunci când se dubleaza sarcina. 

\begin{figure}[H]
\centering
\begin{subfigure}{.45\textwidth}
    \centering
    \includegraphics[width=.95\linewidth]{q5}  
    \caption{Mișcarea pentru o particulă de sarcină q.}
    \label{Sarcină pozitivă}
\end{subfigure}
\begin{subfigure}{.45\textwidth}
    \centering
    \includegraphics[width=.95\linewidth]{q10}  
    \caption{Mișcarea pentru o particulă de sarcină 2q.}
    \label{Sarcină negativă}
\end{subfigure}
\caption{Mișcarea într-un câmp magnetic omogen, ce iese din foaie, a două particule de sarcină diferită.}
\label{mis2}
\end{figure}

\subsection{Utilizarea unui câmp magnetic pentru controlul unei particule încărcate în aproximația 3D}
{
In schimb,în aproximația 3D, dacă particula pătrunde sub un unghi oarecare în câmpul magnetic, atunci 
viteza particulei poate fi descompusă după două direcţii, una paralelă cu linia de câmp magnetic şi 
alta perpendiculară pe linia de câmp. Va exista astfel o mişcare circulară de rază r ca în cazul 2D, combinată cu  o mişcare de translaţie în jurul liniei de câmp. Ca urmare, particula are o 
traiectorie helicoidaă , ca cea reprezentată în figura \ref{helic}.

\begin{figure}[H]
\centering
  \includegraphics[width=0.9\textwidth]{helic}
  \caption{\label{helic}Traiectoria helicoidală a unei particule încarcate in aproxiamția 3D.}
\end{figure}

Folosind această miscare, putem astfel descrie o configurație de câmp magnetic ce poate fi utilizat pentru controlul practic al unei particule încărcate. În fizica plasmei o astfel de configurație este un câmp ce permite fenomenul de refelxie magnetică. Acest fenomen este datorat faptului că forța care acționează asupra unei particule încărcate electric este orientată întotdeauna spre câmpuri mai slabe. Asta înseamnă că, în mișcarea sa, o particulă poate întalni un camp magnetic care să o oprească și să determine întoarcerea acesteia din sensul în care a venit. Acest fenomen, de reflexie magnetică, permite creearea de oglinzi magnetice (vezi figura \ref{oglinda}).

\begin{figure}[H]
\centering
  \includegraphics[width=0.9\textwidth]{oglinda}
  \caption{\label{oglinda}Mișcarea unei particule încarcate într-o oglindă magnetică.}
\end{figure}

Oglinzile magnetice au o puternică întrebuințare în fizica plasmei prin faptul că sunt folosite în realizarea capcanelor magnetice, un dispozitiv vital în confinarea magnetică a plasmei de fuziune. Cea mai simplă capcană magentică se obține folosind două spire de curent apropiate, prin care circulă curenți electrici în același sens (vezi figura \ref{sticla}).

\begin{figure}[H]
\centering
  \includegraphics[width=0.9\textwidth]{sticla}
  \caption{\label{sticla}Ansamblu de spire de curent ce formează o capcană magentică.}
\end{figure}
}


}

{\section{Montajul experimental}

Elementul principal din montajul experimental (vezi \ref{figs}) este un electromagnet ce curbează traiectoria radiațiilor $\beta$ emise de o sursă radioacativă. Intensitatea câmpului magnetic creat de electromagnet poate fi controlată prin modificarea intensitații curentului prin electromagnet  din butonul de la sursa de tensiune. Numaratoare se fa efeta cu un detector Geiger–Müller.

\begin{figure}[H]
\centering
  \includegraphics[width=0.9\textwidth]{montaj.png}
  \caption{\label{figs}Montajul experimental.}
\end{figure}
}


{\section{Modul de lucru}
Pasul 1: Se înregistrează numărul e impulsuri (F) pentru radiația de fond timp de 10 minute ($t_f=10 min =600s$) și se calculează viteza de numărare a fondului cu formula $f=\frac{F}{t_f}$ [imp/s].

Pasul 2: Se introduce sursa de ${}^{90}Sr$ în suportul circular și se cuplează borna pozitivă a sursei de alimentare la borna $\beta^-$ a bobinei.

Pasul 3: Se setează intensitatea curentului brin bobină la prima valoare din tabelul \ref{tab1}.

Pasul 4: Se setează ceasul număratorului la $t= 60s$ și se inregistrează numărul de impulsuri [N], rezultatul se trece în tabelul \ref{tab1}.

Pasul 5: Se calculează viteza de numărare cu formula $n'=\frac{N}{t}$ [imp/s] și separat se scade viteza de numărare a fondului $n=n'-f$.

Pasul 6: Se repetă măsuratorile pentru restul valorilor intensitații curentului prin bobină.

Pasul 7: Se cuplează borna pozitivă a sursei de alimentare la borna $\beta^+$ a bobinei.

Pasul 8: Se parcurg aceleași măsuratori ca și în cazul anterior, iar rezultatele se trec în tabelul \ref{tab2}.
 
}

{\section{Rezultate}

După parcurgerea proceduriilor descrise in sectiunea anterioara au rezultat tabelele \ref{tab1} \ref{tab2} si figuriile \ref{ppp} \ref{ppm}.

Pentru radaiția de fond s-au înregistrat 150 de impusluri în 600 de secunde. Deci $f=0.25 imp/s$.

Privind acum atat la tabelul \ref{tab1} cât și la figura \ref{ppp} observăm că există o valoare a câmpului magnetic pentru care numarul de impulsuri înregistrate este maxim. Această valoare corespunde cu mărimea ideala a câmpului magnetic ce curbează fascicului de radiatii beta pe detctor. O valoare mai mare sau mai mica a câmpului magnetic va produce o deviație a fasciculului de radiații betă înainte sau după detector, acest lucru atrage de la sine valori mai mici ale numarului de impulsuri înregistrate.

Ne concentrăm atenția pe tabelul \ref{tab2} cât și la figura \ref{ppm} observăm că valorile numărului de impulsuri nu mai sunt dependente de valaorea intensitații câmpului magnetic. Acest lucru se întâmpla deoarece am shimbat bornele sursei deci și sensul câmpului magnetic. Deci acesta va curba traiectoria radiațiilor beta în sensul opus detectorului, astfel vom înregistra doar radiația de fond. În realitate se întregistrează un numar de impulsuri foarte apropiat de radiația de fond deoarece unele particule încărcate tot reușesc să ajungă pe detector deoarece au o energie ridicată și nu pot fi deviate în timp de câmpul magentic. Acest lucru este evident în special pentru valori mici ale câmpului magnetic.

\begin{table}[H]
\begin{center}
\begin{tabular}{|c|c|c|c|c|c|c|} 
 \hline
 Nr. crt & I [A] & b[mT]  &E[keV] & N(imp) &  n' [Imp/s] & n [imp/s] \\ 
 \hline\hline
1 & 0 & 4.4 & 5.47 &  155 & 2.58 &2.33 \\
 \hline
 2 & 0.1 & 15.4 & 21.56 &  258 & 4.3 & 4.05  \\
 \hline
 3 & 0.2 &  24.5 & 47.34 & 336 & 5.6 & 5.35 \\
 \hline
 4 & 0.3 &34.7 & 81.55 &534& 8.9&8.65 \\
 \hline
 5 & 0.4 & 45.7 & 122.83 &789 &13.15 & 12.9\\
 \hline
 6 & 0.5 & 56.1 & 169.89 &1008 & 16.8 &16.55 \\
 \hline
 7 & 0.6 & 65.8 & 221.62 &1140 & 19 &18.75 \\
 \hline
 8 & 0.7 & 78 & 277.11 & 1267 & 21.11 &20.86 \\
 \hline
 9 & 0.8 & 87 & 335.61 & 1296 & 21.6 &21.35 \\
 \hline
 10 & 0.9 & 97.4 & 396.53 & 1278 & 21.3 &21.05 \\
 \hline
 11 & 1.0 & 107.4 & 459.43 &1076 & 17.93 &17.68 \\
 \hline
 12 & 1.1 & 120.2 & 523.94 &1004 & 16.73 &16.48 \\
 \hline
 13 & 1.2 & 128.5 & 589.79 &894 & 14.9 &12.65 \\
 \hline
 14 & 1.3 & 140 & 656.74 &735 & 12.25 &12 \\
 \hline
 15 & 1.4 & 149 & 724.61 &618 & 10.3 &10.05 \\
 \hline
 16 & 1.5 & 159.3 & 793.27 &477 & 7.95 &7.7 \\
 \hline
 17 & 1.6 & 168.1 & 861.58 &412 & 6.86 &6.61 \\
 \hline
 18 & 1.7 & 174.7 & 932.47 &308 & 5.13 &4.88 \\
 \hline
\end{tabular}
\caption{\label{tab1}Datele colectate pentru sursa de ${}^{90}Sr$ atunci când borna pozitivă a sursei este conectată la borna negativă a sursei.}
\end{center}
\end{table}


\begin{table}[H]
\begin{center}
\begin{tabular}{|c|c|c|c|c|c|c|} 
 \hline
 Nr. crt & I [A] & b[mT]  &E[keV] & N[imp] &  n' [Imp/s] & n [imp/s] \\ 
 \hline\hline
1 & 0 & 4.4 & 5.47 &  82 & 1.36 &1.11 \\
 \hline
 2 & 0.1 & 15.4 & 21.56 &  81 & 1.35 & 1.1  \\
 \hline
 3 & 0.2 &  24.5 & 47.34 & 70 & 1.16 & 0.91 \\
 \hline
 4 & 0.3 &34.7 & 81.55 &57& 0.95&0.7 \\
 \hline
 5 & 0.4 & 45.7 & 122.83 &43 &0.71 & 0.46\\
 \hline
 6 & 0.5 & 56.1 & 169.89 &38 & 0.63 &0.38 \\
 \hline
 7 & 0.6 & 65.8 & 221.62 &31 & 0.51 &0.26 \\
 \hline
 8 & 0.7 & 78 & 277.11 & 34 & 0.56 &0.31 \\
 \hline
 9 & 0.8 & 87 & 335.61 & 34 & 0.56 &0.31 \\
 \hline
 10 & 0.9 & 97.4 & 396.53 & 24 & 0.4 &0.15 \\
 \hline
 11 & 1.0 & 107.4 & 459.43 &30 & 0.5 &0.25 \\
 \hline
 12 & 1.1 & 120.2 & 523.94 &36 & 0.6 &0.35 \\
 \hline
 13 & 1.2 & 128.5 & 589.79 &19 & 0.31 &0.06 \\
 \hline
 14 & 1.3 & 140 & 656.74 &20 & 0.33 &0.08\\
 \hline
 15 & 1.4 & 149 & 724.61 &15 & 0.25 &0 \\
 \hline
 16 & 1.5 & 159.3 & 793.27 &20 & 0.33 &0.08 \\
 \hline
 17 & 1.6 & 168.1 & 861.58 &16 & 0.26 &0.01 \\
 \hline
 18 & 1.7 & 174.7 & 932.47 &15 & 0.25 &0 \\
 \hline
\end{tabular}
\caption{\label{tab2}Datele colectate pentru sursa de ${}^{90}Sr$ atunci când borna pozitivă a sursei este conectată la borna pozitivă a sursei.}
\end{center}
\end{table}



\begin{figure}[H]
\centering
  \includegraphics[width=0.9\textwidth]{pm.png}
  \caption{\label{ppm}Graficul distribuției după energie pentru sursa de ${}^{90}Sr$ atunci când borna pozitivă a sursei este conectată la borna $\beta^-$ a sursei.}
\end{figure}

\begin{figure}[H]
\centering
  \includegraphics[width=0.9\textwidth]{pp.png}
  \caption{\label{ppp}Graficul distribuției după energie pentru sursa de ${}^{90}Sr$ atunci când borna pozitivă a sursei este conectată la borna $\beta^+$ a sursei.}
\end{figure}

}

{\section{Controlul traiectoriei particulelor încărcate}
Montajul experimental descris în secțiunea 3 poate fi utilizat pentru controlul traiectoriei particulelor încărcate. Vom arăta acest lucru prin utilizarea unui simulator.

Considerăm o particulă cu masa, sarcină si viteză inițială constantă, putem controla traiectoria acesteia prin trecerea acesteia print-un câmp magnetic.
În urma utilizarii simulatorului (vezi \ref{FIGU}) observăm faptul că un câmp magnetic mai puternic va devia mai mult traiectoria unei particule încarcate i daca directia campului magnetic se schimba, atunci si deviatia are loc in sensul opus.

\begin{figure}[H]
\centering
\begin{subfigure}{.95\textwidth}
    \centering
    \includegraphics[width=.95\linewidth]{tesl}  
    \caption{}
    \label{SUBFIGURE LABEL 1}
\end{subfigure}
\caption{Dependența traiectoriei unei particule încărcate de intensitatea câmpului magnetic prin care trece.}
\label{FIGU}
\end{figure}

In urma utilizarii simulatorului realizam faptul ca un camp magnetic mai puternic va devia mai mult traiectoria unei particule incarcate și dacă direcția câmpului magnetic se schimbaă, atunci și deviația are loc în sensul opus.

}





{\section{Concluzii}
}

\printbibliography

\end{document}