\documentclass[12pt]{article}   %12 point font for Times New Roman


\usepackage{xcolor}
\usepackage{color}
\usepackage{subcaption}
\usepackage{float}
\usepackage{mathrsfs}
\usepackage{revsymb}
\usepackage{booktabs}
\usepackage{multirow}
\usepackage{rotating}
\usepackage{amssymb}
\usepackage{nicefrac}
\usepackage{graphicx}
\usepackage{epsfig}
\usepackage{mathrsfs}
\usepackage{xcolor}
\usepackage{amsmath}
\usepackage{amsthm}
\usepackage{graphicx}  %for images and plots
\usepackage[a4paper, left=2.5cm, right=2.5cm, top=2cm, bottom=2cm]{geometry}
\usepackage{setspace}  %use this package to set linespacing as desired
\usepackage{times}  %set Times New Roman as the font
\usepackage[explicit]{titlesec}  %title control and formatting

\usepackage[titles]{tocloft}  %table of contents control and formatting
\usepackage[backend=bibtex, sorting=none, bibstyle=ieee]{biblatex}  %reference manager
\usepackage[bookmarks=true, hidelinks]{hyperref}
\usepackage[page]{appendix}  %for appendices
\usepackage{rotating}  %for rotated, landscape images
\usepackage[normalem]{ulem}  %for italicized text
\usepackage{tabto}

%\newcommand{\sectionbreak}{\clearpage}

\usepackage[utf8]{inputenc}

\usepackage{amsmath}
\usepackage{verbatim}
\usepackage{hyperref}
\usepackage{url}
\usepackage{pdfcomment}
\usepackage{multirow}
\usepackage{gensymb}
\usepackage{array}

\usepackage[export]{adjustbox}

\usepackage[toc]{glossaries}

\makeglossaries

\usepackage{afterpage}

%\newcommand\blankpage{%
%	\null
%	\thispagestyle{empty}%
%	\addtocounter{page}{-1}%
%	\newpage}

%TABLE STUFF
%END TABLE STUFF

\DeclareUnicodeCharacter{2212}{-}

\DeclareUnicodeCharacter{223C}{~}

\makeatletter
\newcommand\footnoteref[1]{\protected@xdef\@thefnmark{\ref{#1}}\@footnotemark}
\makeatother

\makeatletter
\newcounter{subsubparagraph}[subparagraph]
\renewcommand\thesubsubparagraph{%
	\thesubparagraph.\@arabic\c@subsubparagraph}
\newcommand\subsubparagraph{%
	\@startsection{subsubparagraph}    % counter
	{6}                              % level
	{\parindent}                     % indent
	{3.25ex \@plus 1ex \@minus .2ex} % beforeskip
	{-1em}                           % afterskip
	{\normalfont\normalsize\bfseries}}
\newcommand\l@subsubparagraph{\@dottedtocline{6}{10em}{5em}}
\newcommand{\subsubparagraphmark}[1]{.}
\makeatother

\usepackage[english]{babel}




\makeatletter
\providecommand\add@text{}
\newcommand\tagaddtext[1]{%
  \gdef\add@text{#1\gdef\add@text{}}}% 
\renewcommand\tagform@[1]{%
  \maketag@@@{\llap{\add@text\quad}(\ignorespaces#1\unskip\@@italiccorr)}%
}
\makeatother



\newtheorem{theorem}{Addendum}
%\newlength{\overwritelength}
%\newlength{\minimumoverwritelength}
%\setlength{\minimumoverwritelength}{1cm}
%\newcommand{\overwrite}[3][red]{%
%	\settowidth{\overwritelength}{$#2$}%
%	\ifdim\overwritelength<\minimumoverwritelength%
%	\setlength{\overwritelength}{\minimumoverwritelength}\fi%
%	\stackrel
%	{%
%		\begin{minipage}{\overwritelength}%
%			\color{#1}\centering\small #3\\%
%			\rule{1pt}{9pt}%
%	\end{minipage}}
%	{\colorbox{#1!50}{\color{black}$\displaystyle#2$}}}

\newlength{\overwritelength}
\newlength{\minimumoverwritelength}
\setlength{\minimumoverwritelength}{1cm}
\newcommand{\overwrite}[3][red]{%
	\settowidth{\overwritelength}{$#2$}%
	\ifdim\overwritelength<\minimumoverwritelength%
	\setlength{\overwritelength}{\minimumoverwritelength}\fi%
	\stackrel
	{%
		\begin{minipage}{\overwritelength}%
			\color{#1}\centering\small #3\\%
			\rule{1pt}{9pt}%
	\end{minipage}}
	{\colorbox{#1!50}{\color{black}$\displaystyle#2$}}}

%%%%%%%%%%%%%%%%%%%%%%%%%%%%%%%%%%%
% Bibliography
%%%%%%%%%%%%%%%%%%%%%%%%%%%%%%%%%%%

%Add your bibliography file here
\bibliography{refc1}


% prevent certain fields in references from printing in bibliography
\AtEveryBibitem{\clearfield{issn}}
\AtEveryBibitem{\clearlist{issn}}

\AtEveryBibitem{\clearfield{language}}
\AtEveryBibitem{\clearlist{language}}

\AtEveryBibitem{\clearfield{doi}}
\AtEveryBibitem{\clearlist{doi}}

\AtEveryBibitem{\clearfield{url}}
\AtEveryBibitem{\clearlist{url}}

\AtEveryBibitem{%
	\ifentrytype{online}
	{}
	{\clearfield{urlyear}\clearfield{urlmonth}\clearfield{urlday}}}

%%%%%%%%%%%%%%%%%%%%%%%%%%%%%%%%%%
%paragraph indent
%%%%%%%%%%%%%%%%%%%%%%%%%%%%%%%%%%

\setlength{\parindent}{4em}
\setlength{\parskip}{1em}

%%%
%The subsubsub...subsection format
%%%


\title{Generarea sarcinilor electrice cu ajutorul radiațiilor X}
\author{Ștefan-Răzvan~Anton\\ Anul 3, Grupa 1334,\\ Facultatea de Științe Aplicate}

\begin{document}

\maketitle




{\section{Scopul lucrării}
%
1. Documentarea și înțelegera modului de funcționare a globului de plasmă.\\
2. Descrierea principalelor componente ale acestuia ce îi asigură funcționarea.\\
3. Măsurarea și analiza parametrilor de funcționare ai sistemului.
}
%

{\section{Introducere}

}

{\section{Montajul experimental}
Privind în partea de sus a globului de plasmă, se observă că acesta este alcătuit dintr-o incintă închisă de scticlă, umplută cu un amestec de diverse gaze nobile, cum ar fi Argon, Xenon, Neon și Kripton. În centrul globului de sticlă se găseste un electrod de sticlă cu acoperire interioară de grafit, înăuntrul căruia se află un strat de lână de oțel conectată la un curent de înaltă frecvență. Privind acum în partea de jos a globului de plasmă se obseră o placuță metalică ce va fi folosită drept punct de referință pentru măsurarea potențialului. Se observă de asemenea și partea de alimentare electrică a globului de plasmă, acesta poate fi alimentat printr-un cablu USB sau prin introducerea a trei baterii de tip AAA.
 
\begin{figure}[H]
    \centering
    \subfloat[\centering Vedere de sus a globului de plasmă]{{\includegraphics[width=7cm]{gsus} }}%
    \qquad
    \subfloat[\centering Vedere de jos a globului de plasmă]{{\includegraphics[width=7cm]{gjos} }}%
    \caption{Elementele globului de plasmă}%
    \label{fig:example}%
\end{figure}
}


{\section{Principiul de funcționare}
În figura \ref{circ} observăm schematica internă a circuitului ce alimentează electrodul de sticlă. 
Circuitul funcționează în modul următor:
Tensiunea de alimentare de 5V (care provine de la un cablu USB sau de la trei baterii de tip AAA) este atenuată de capacitorul "C". Partea imprejmuită cu rosu alcatuiește un circuit oscilant, al cărui frecventă de oscilație este dictată de frecvența de ocscilație a inductorului numit "Reacție", acesta, la rândul lui, având aceesi frecveență cu frecvența proprie a inductorului numit "Primar". Între inductorii "Primar" și "Secundar" are loc fenomenul de inducție electromagnetică. Astfel, în bucla "Secundar" regăsim o tensiune de câțiva mii de volți la o frecvența de până la câțiva zeci de kilohertzi.



\begin{figure}[H]
\centering
  \includegraphics[width=0.9\textwidth]{circ}
  \caption{\label{circ}Schematica internă a globului de plasmă.}
\end{figure}
}

{\section{Modul de lucru}
Pentru a putea descrie potențialul electric de pe suprafața globului de plasmă vom urma proceda astfel:

Pasul 1: Marcăm câteva puncte de test pe suprafața globului de plasmă(figura \ref{pct}) acestea sunt punctele în care se va măsura potențialul.

Pasul 2: Un braț al multimetrului se va atașa de placuța metalică de la bază globului de plasmă \ref{fig:example}, acesta va fi punctul de referință pentru toate măsuratoriile urmatoare.

Pasul 3: Celălalt braț al multimetrului se va pune pe rănd în punctele marcate la pasul 1 și se va citi tensiunea afișată pe multimetru.

Pasul 4: Se repetă măsuratoriile de 10 ori, iar media măsurătoriilor se trece intr-un tabel.

Pasul 5: Tinănd cont de reprezentarea 2D a punctelor în care a fost măsurat potențialul (vezi fig\ref{princ}), se realizează o reprezentare 2D a potențialului de pe sferă.

\begin{figure}[H]
\centering
  \includegraphics[width=0.9\textwidth]{sfera}
  \caption{\label{pct}Punctele de pe globul de plasmă în care a fost măsurat potențialul.}
\end{figure}

\begin{figure}[H]
\centering
  \includegraphics[width=0.9\textwidth]{proiectie}
  \caption{\label{princ}Reprezentarea 2D a globului de plasmă împreună cu punctele în care a fost măsurat potențialul.}
\end{figure}
}

{\section{Rezultate}

În urma măsurătoriilor a rezultat tabelul de valori \ref{tab1} și figura \ref{fig:ex}.
Se observă că odată cu coborârea pe globul de plasmă difernța de potențial tinde să crească. Ne așteptam la acest femonem deoarece distanța dintre electrodul de sticlă si proba multimetrului crește, deci și diferența de potențial dintre masă și punctul în care se face măsurătoarea crește.

O mențiune importantă este ca în distribuția potențialului pe suprafață sferei de sticlă există o asimetrie datorată poziționării masei aproape de punctul jos față (lucru vizibil in figuriile \ref{pct} si ref{princ}). Astfel la poziționarea brațului de măsură al multimetrului apare fasciculul de plasmă "se scurge" spre masă (vezi figura \ref{scr}).
 
\begin{table}[H]
\begin{center}
\begin{tabular}{|c|c|} 
 \hline
 Punct de pe glob & U[V]   \\ 
 \hline\hline
sus & 4.22  \\
 \hline
 stânga & 5.61 \\
 \hline
 dreapta & 6.65  \\
 \hline
 spate & 10.98  \\
 \hline
 față & 4.11  \\
 \hline
 jos stânga & 6.77  \\
 \hline
 jos dreapta & 9.64  \\
 \hline
 jos spate & 15.74   \\
 \hline
 jos față & 5.33  \\
 \hline
\end{tabular}
\caption{\label{tab1}Media a 10 măsuratori pentru potențialul electric măsurat în diferite puncte ale globului de plasmă.}
\end{center}
\end{table}

\begin{figure}[H]
    \centering
    \subfloat[\centering Harta 2D a potențialului de glob.]{{\includegraphics[width=7cm]{rez1} }}%
    \qquad
    \subfloat[\centering Interpolarea valoriilor potențialului de pe glob.]{{\includegraphics[width=7cm]{rez inter} }}%
    \caption{Reprezentarea 2D a potentialului de pe sferă.}%
    \label{fig:ex}%
\end{figure}


\begin{figure}[H]
\centering
  \includegraphics[width=0.9\textwidth]{scr}
  \caption{\label{scr}Fenomenul de "scurgere" a fasciculului de plasă spre masă la apropierea brațului de măsura al multimetrului de punctul jos față.}
\end{figure}
}







{\section{Concluzii}
În urma analizei modului de functionare al unui glob de plasmă am descris modul în care se crează o tensiunea de frecvență și voltaj mare pe electrodul de sticlă dinauntrul globului de plasmă umplut cu un gaz inert. Am aflat distribuția potențialului pe suprafața sferei de sticlă prin interpolarea valoriilor măsurate in diverse puncte. Am arătat faptul că există o asimetrie în distribuția de potential datorată poziționării masei dinauntrul globului de sticlă.
}

\printbibliography

\end{document}