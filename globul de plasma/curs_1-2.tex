\documentclass[12pt]{article}   %12 point font for Times New Roman


\usepackage{xcolor}
\usepackage{color}
\usepackage{subcaption}
\usepackage{float}
\usepackage{mathrsfs}
\usepackage{revsymb}
\usepackage{booktabs}
\usepackage{multirow}
\usepackage{rotating}
\usepackage{amssymb}
\usepackage{nicefrac}
\usepackage{graphicx}
\usepackage{epsfig}
\usepackage{mathrsfs}
\usepackage{xcolor}
\usepackage{amsmath}
\usepackage{amsthm}
\usepackage{graphicx}  %for images and plots
\usepackage[a4paper, left=2.5cm, right=2.5cm, top=2cm, bottom=2cm]{geometry}
\usepackage{setspace}  %use this package to set linespacing as desired
\usepackage{times}  %set Times New Roman as the font
\usepackage[explicit]{titlesec}  %title control and formatting

\usepackage[titles]{tocloft}  %table of contents control and formatting
\usepackage[backend=bibtex, sorting=none, bibstyle=ieee]{biblatex}  %reference manager
\usepackage[bookmarks=true, hidelinks]{hyperref}
\usepackage[page]{appendix}  %for appendices
\usepackage{rotating}  %for rotated, landscape images
\usepackage[normalem]{ulem}  %for italicized text
\usepackage{tabto}

%\newcommand{\sectionbreak}{\clearpage}

\usepackage[utf8]{inputenc}

\usepackage{amsmath}
\usepackage{verbatim}
\usepackage{hyperref}
\usepackage{url}
\usepackage{pdfcomment}
\usepackage{multirow}
\usepackage{gensymb}
\usepackage{array}

\usepackage[export]{adjustbox}

\usepackage[toc]{glossaries}

\makeglossaries

\usepackage{afterpage}

%\newcommand\blankpage{%
%	\null
%	\thispagestyle{empty}%
%	\addtocounter{page}{-1}%
%	\newpage}

%TABLE STUFF
%END TABLE STUFF

\DeclareUnicodeCharacter{2212}{-}

\DeclareUnicodeCharacter{223C}{~}

\makeatletter
\newcommand\footnoteref[1]{\protected@xdef\@thefnmark{\ref{#1}}\@footnotemark}
\makeatother

\makeatletter
\newcounter{subsubparagraph}[subparagraph]
\renewcommand\thesubsubparagraph{%
	\thesubparagraph.\@arabic\c@subsubparagraph}
\newcommand\subsubparagraph{%
	\@startsection{subsubparagraph}    % counter
	{6}                              % level
	{\parindent}                     % indent
	{3.25ex \@plus 1ex \@minus .2ex} % beforeskip
	{-1em}                           % afterskip
	{\normalfont\normalsize\bfseries}}
\newcommand\l@subsubparagraph{\@dottedtocline{6}{10em}{5em}}
\newcommand{\subsubparagraphmark}[1]{.}
\makeatother

\usepackage[english]{babel}




\makeatletter
\providecommand\add@text{}
\newcommand\tagaddtext[1]{%
  \gdef\add@text{#1\gdef\add@text{}}}% 
\renewcommand\tagform@[1]{%
  \maketag@@@{\llap{\add@text\quad}(\ignorespaces#1\unskip\@@italiccorr)}%
}
\makeatother



\newtheorem{theorem}{Addendum}
%\newlength{\overwritelength}
%\newlength{\minimumoverwritelength}
%\setlength{\minimumoverwritelength}{1cm}
%\newcommand{\overwrite}[3][red]{%
%	\settowidth{\overwritelength}{$#2$}%
%	\ifdim\overwritelength<\minimumoverwritelength%
%	\setlength{\overwritelength}{\minimumoverwritelength}\fi%
%	\stackrel
%	{%
%		\begin{minipage}{\overwritelength}%
%			\color{#1}\centering\small #3\\%
%			\rule{1pt}{9pt}%
%	\end{minipage}}
%	{\colorbox{#1!50}{\color{black}$\displaystyle#2$}}}

\newlength{\overwritelength}
\newlength{\minimumoverwritelength}
\setlength{\minimumoverwritelength}{1cm}
\newcommand{\overwrite}[3][red]{%
	\settowidth{\overwritelength}{$#2$}%
	\ifdim\overwritelength<\minimumoverwritelength%
	\setlength{\overwritelength}{\minimumoverwritelength}\fi%
	\stackrel
	{%
		\begin{minipage}{\overwritelength}%
			\color{#1}\centering\small #3\\%
			\rule{1pt}{9pt}%
	\end{minipage}}
	{\colorbox{#1!50}{\color{black}$\displaystyle#2$}}}

%%%%%%%%%%%%%%%%%%%%%%%%%%%%%%%%%%%
% Bibliography
%%%%%%%%%%%%%%%%%%%%%%%%%%%%%%%%%%%

%Add your bibliography file here
\bibliography{refc1}


% prevent certain fields in references from printing in bibliography
\AtEveryBibitem{\clearfield{issn}}
\AtEveryBibitem{\clearlist{issn}}

\AtEveryBibitem{\clearfield{language}}
\AtEveryBibitem{\clearlist{language}}

\AtEveryBibitem{\clearfield{doi}}
\AtEveryBibitem{\clearlist{doi}}

\AtEveryBibitem{\clearfield{url}}
\AtEveryBibitem{\clearlist{url}}

\AtEveryBibitem{%
	\ifentrytype{online}
	{}
	{\clearfield{urlyear}\clearfield{urlmonth}\clearfield{urlday}}}

%%%%%%%%%%%%%%%%%%%%%%%%%%%%%%%%%%
%paragraph indent
%%%%%%%%%%%%%%%%%%%%%%%%%%%%%%%%%%

\setlength{\parindent}{4em}
\setlength{\parskip}{1em}

%%%
%The subsubsub...subsection format
%%%


\title{Generarea sarcinilor electrice cu ajutorul radiațiilor X}
\author{Ștefan-Răzvan~Anton\\ Anul 3, Grupa 1334,\\ Facultatea de Științe Aplicate}

\begin{document}

\maketitle




{\section{Scopul lucrării}
%
1.Documentarea și înțelegera modului de funcționare a globului de plasmă.\\
2. Descrierea principalelor componente ale acestuia ce îi asigură funcționarea.\\
3. Măsurarea și analiza parametrilor de funcționare ai sistemului.
}
%


{\section{Montajul experimental}
Privint în partea de sus a globului de plasmă , se observă că acesta este alcătuit dintr-o incintă închisă de scticlă, umplută cu un amestec de diverse gaze nobile, cu mar fi argon, xenon, neon și kripton. În centrul gobului de sticlă se găseste un electrod de sticlă cu acoperire interioară de grafit, înăuntul căruia se află un strat de lână de oțel conectată la un curent de înaltă frecventă. Privint acum în partea de jos a globului de plasm-a se obseră placuța metalică ce va fi folosită drept punct de referința pentru masurătoiile ce vor urma. Se observă de asemenea și partea de alimentare electrică a globului de plasmă, acesta poate fi alimentat printr-un cablu USB sau prin introducerea a trei baterii de tip AAA.
 
\begin{figure}[H]
    \centering
    \subfloat[\centering Vedere de sus a globului de plasmă]{{\includegraphics[width=7cm]{gsus} }}%
    \qquad
    \subfloat[\centering Vedere de jos a globului de plasmă]{{\includegraphics[width=7cm]{gjos} }}%
    \caption{2 Figures side by side}%
    \label{fig:example}%
\end{figure}
}


{\section{Principiul de funcționare}
În figura \ref{circ} observăm schematica internă a circuitului ce alimentează electrodul de sticlă. 
Circuitul funcționează în modul următor: 
Curentul de alimentare de 5V (care provine de la un cablu USB sau de la trei baterii de tip AAA) 
Astfel în bucla "secondary" regăsim o tensiune de câțiva mii de volți la o frecvența de până la câțiva zeci de kilohertzi.



\begin{figure}[H]
\centering
  \includegraphics[width=0.9\textwidth]{circ}
  \caption{\label{circ}Schematica internă a globului de plasmă.}
\end{figure}
}

{\section{Modul de lucru}
\begin{figure}[H]
\centering
  \includegraphics[width=0.9\textwidth]{sfera}
  \caption{\label{princ}Punctele de pe golbul de plasmă în care a fost măsurat potențialul.}
\end{figure}

\begin{figure}[H]
\centering
  \includegraphics[width=0.9\textwidth]{proiectie}
  \caption{\label{princ}reprezentarea 2D a globului de plasmă împreună cu punctele în care a fost măsurat potențialul.}
\end{figure}
}

{\section{Rezultate}

\begin{table}[H]
\begin{center}
\begin{tabular}{|c|c|c|c|} 
 \hline
 Nr. crt & I [mA] & U[V]  &$I_c$[nA]  \\ 
 \hline\hline
1 & 0 & 0 & 0  \\
 \hline
 2 & 0.1 & 0.61 &  0.61  \\
 \hline
 3 & 0.2 &  1.15 &  1.15  \\
 \hline
 4 & 0.3 & 1.57 &1.57 \\
 \hline
 5 & 0.4 & 1.96 & 1.96  \\
 \hline
 6 & 0.5 & 2.20 & 2.20  \\
 \hline
 7 & 0.6 & 2.64 & 2.64  \\
 \hline
 8 & 0.7 & 2.96 & 2.96  \\
 \hline
 9 & 0.8 & 3.20 & 3.20 \\
 \hline
 10 & 0.9 & 3.71 & 3.71  \\
 \hline
 11 & 1.0 & 4.06 & 4.06  \\
 \hline
\end{tabular}
\caption{\label{tab1}Curentul de ionizare de saturație $I_c$ ca funcție de curentul de emisie I.}
\end{center}
\end{table}

}







{\section{Concluzii}
}

\printbibliography

\end{document}