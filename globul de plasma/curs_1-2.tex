\documentclass[12pt]{article}   %12 point font for Times New Roman


\usepackage{xcolor}
\usepackage{color}
\usepackage{subcaption}
\usepackage{float}
\usepackage{mathrsfs}
\usepackage{revsymb}
\usepackage{booktabs}
\usepackage{multirow}
\usepackage{rotating}
\usepackage{amssymb}
\usepackage{nicefrac}
\usepackage{graphicx}
\usepackage{epsfig}
\usepackage{mathrsfs}
\usepackage{xcolor}
\usepackage{amsmath}
\usepackage{amsthm}
\usepackage{graphicx}  %for images and plots
\usepackage[a4paper, left=2.5cm, right=2.5cm, top=2cm, bottom=2cm]{geometry}
\usepackage{setspace}  %use this package to set linespacing as desired
\usepackage{times}  %set Times New Roman as the font
\usepackage[explicit]{titlesec}  %title control and formatting

\usepackage[titles]{tocloft}  %table of contents control and formatting
\usepackage[backend=bibtex, sorting=none, bibstyle=ieee]{biblatex}  %reference manager
\usepackage[bookmarks=true, hidelinks]{hyperref}
\usepackage[page]{appendix}  %for appendices
\usepackage{rotating}  %for rotated, landscape images
\usepackage[normalem]{ulem}  %for italicized text
\usepackage{tabto}

%\newcommand{\sectionbreak}{\clearpage}

\usepackage[utf8]{inputenc}

\usepackage{amsmath}
\usepackage{verbatim}
\usepackage{hyperref}
\usepackage{url}
\usepackage{pdfcomment}
\usepackage{multirow}
\usepackage{gensymb}
\usepackage{array}

\usepackage[export]{adjustbox}

\usepackage[toc]{glossaries}

\makeglossaries

\usepackage{afterpage}

%\newcommand\blankpage{%
%	\null
%	\thispagestyle{empty}%
%	\addtocounter{page}{-1}%
%	\newpage}

%TABLE STUFF
%END TABLE STUFF

\DeclareUnicodeCharacter{2212}{-}

\DeclareUnicodeCharacter{223C}{~}

\makeatletter
\newcommand\footnoteref[1]{\protected@xdef\@thefnmark{\ref{#1}}\@footnotemark}
\makeatother

\makeatletter
\newcounter{subsubparagraph}[subparagraph]
\renewcommand\thesubsubparagraph{%
	\thesubparagraph.\@arabic\c@subsubparagraph}
\newcommand\subsubparagraph{%
	\@startsection{subsubparagraph}    % counter
	{6}                              % level
	{\parindent}                     % indent
	{3.25ex \@plus 1ex \@minus .2ex} % beforeskip
	{-1em}                           % afterskip
	{\normalfont\normalsize\bfseries}}
\newcommand\l@subsubparagraph{\@dottedtocline{6}{10em}{5em}}
\newcommand{\subsubparagraphmark}[1]{.}
\makeatother

\usepackage[english]{babel}




\makeatletter
\providecommand\add@text{}
\newcommand\tagaddtext[1]{%
  \gdef\add@text{#1\gdef\add@text{}}}% 
\renewcommand\tagform@[1]{%
  \maketag@@@{\llap{\add@text\quad}(\ignorespaces#1\unskip\@@italiccorr)}%
}
\makeatother



\newtheorem{theorem}{Addendum}
%\newlength{\overwritelength}
%\newlength{\minimumoverwritelength}
%\setlength{\minimumoverwritelength}{1cm}
%\newcommand{\overwrite}[3][red]{%
%	\settowidth{\overwritelength}{$#2$}%
%	\ifdim\overwritelength<\minimumoverwritelength%
%	\setlength{\overwritelength}{\minimumoverwritelength}\fi%
%	\stackrel
%	{%
%		\begin{minipage}{\overwritelength}%
%			\color{#1}\centering\small #3\\%
%			\rule{1pt}{9pt}%
%	\end{minipage}}
%	{\colorbox{#1!50}{\color{black}$\displaystyle#2$}}}

\newlength{\overwritelength}
\newlength{\minimumoverwritelength}
\setlength{\minimumoverwritelength}{1cm}
\newcommand{\overwrite}[3][red]{%
	\settowidth{\overwritelength}{$#2$}%
	\ifdim\overwritelength<\minimumoverwritelength%
	\setlength{\overwritelength}{\minimumoverwritelength}\fi%
	\stackrel
	{%
		\begin{minipage}{\overwritelength}%
			\color{#1}\centering\small #3\\%
			\rule{1pt}{9pt}%
	\end{minipage}}
	{\colorbox{#1!50}{\color{black}$\displaystyle#2$}}}

%%%%%%%%%%%%%%%%%%%%%%%%%%%%%%%%%%%
% Bibliography
%%%%%%%%%%%%%%%%%%%%%%%%%%%%%%%%%%%

%Add your bibliography file here
\bibliography{refc1}


% prevent certain fields in references from printing in bibliography
\AtEveryBibitem{\clearfield{issn}}
\AtEveryBibitem{\clearlist{issn}}

\AtEveryBibitem{\clearfield{language}}
\AtEveryBibitem{\clearlist{language}}

\AtEveryBibitem{\clearfield{doi}}
\AtEveryBibitem{\clearlist{doi}}

\AtEveryBibitem{\clearfield{url}}
\AtEveryBibitem{\clearlist{url}}

\AtEveryBibitem{%
	\ifentrytype{online}
	{}
	{\clearfield{urlyear}\clearfield{urlmonth}\clearfield{urlday}}}

%%%%%%%%%%%%%%%%%%%%%%%%%%%%%%%%%%
%paragraph indent
%%%%%%%%%%%%%%%%%%%%%%%%%%%%%%%%%%

\setlength{\parindent}{4em}
\setlength{\parskip}{1em}

%%%
%The subsubsub...subsection format
%%%


\title{Generarea sarcinilor electrice cu ajutorul radiațiilor X}
\author{Ștefan-Răzvan~Anton\\ Anul 3, Grupa 1334,\\ Facultatea de Științe Aplicate}

\begin{document}

\maketitle




{\section{Scopul lucrării}
%
1.Prezentarea unei metode de punere în evidență a generării de sarcini electrice în aer cu ajutorul radiațiilor X.\\
2. Analiza influenței parametriilor de funcționare a sursei de raze X asupra numărului de sarcini electrice create.
}
%
{\section{Principiul fizic}
Radiațiile X se detectează pe baza efectelor fizice pe care le produc. De exemplu, înegrirea filmelor fotografice, ionizarea aerului și a altor gaze, producerea efectului fotoelectric la suprafața metalelor sau producerea efectului de luminiscentă în unele substanțe fluorescente.
În contextul acestei lucrări, generearea de sarcini electrice în aer cu ajutorul radiațiilor X poate fi pusă în evidență prin măsurarea curentului de ionizare dintr-un condensator plan cu aer.
Când se aplică o tensiune $U_c$ pe placiile condensatorului, perechiile electron-ion generate sub acțiunea radiațiilor X de energie $\hbar \omega$ sunt colectate pe plăcile condensatorului (vezi figura \ref{princ}). Curentul generat în acest mod corespunde curentului de ionizare $I_c$.
Cu cât tensiunea $U_c$ crește, cu atât și numărul de purtători de sarcină colectați pe plăcile condensatorului crește. După un anumit punct creșterea tensiunii $U_c$ nu mai are ca efect creșterea curentului $I_c$ deoarece acesta a ajuns la saturație, fiind capturați toți purtătorii de sarcină formați de radiația incidentă pe unitatea de timp.
\begin{figure}[H]
\centering
  \includegraphics[width=0.9\textwidth]{princip}
  \caption{\label{princ}Principiul fizic prin care putem pune în evidență generarea de sarcini electrice în aer cu ajutorul radiațiilor X.}
\end{figure}

}

{\section{Montajul experimental}
Radiațiile X ies prin orificiul 1, mai apoi trec prin fanta de colimare 2 și intră între armărutiile condensatorului cu aer 3,4. Radiația este oprită în drumul său de către capacul cutiei 5 care are rol de protecție. Curentul de ionizare cules trece printr-o rezistență de $1G\ohm$, iar tensiunea rezultantă este amplificată și masurată de voltmetrul 8. Sursa 7 produce atât tensiunea pentru alimentarea amplificatorului cât și tensiunea de pe placile condensatorului care este masurată cu voltmetrul 9.

\begin{figure}[H]
\centering
  \includegraphics[width=0.9\textwidth]{montaj}
  \caption{\label{princ}Montajul experimental.}
\end{figure}
}


{\section{Modul de lucru}

Pentru măsurarea curentului de ionizare $I_c$ ca funcție de curentul de emisie $I$ se parcurg pașii:

Pasul 1: Se setează tensiunea tubului la valoarea de $35kV$.\\
Pasul 2: Se setează tensiunea condensatorului la o valoare mai mare de $140V$, astfel încât sa se obtină saturația curentului de ionizare $I_c$.\\
Pasul 3: Se marește curentul de emisie al tubului $I$ în trepte de $0.1 mA$ începand de la $0 mA $ pană la $1 mA$ și se determină curentul de ionizare corespunzător pentru fiecare valoare.\\
Pasul 4: Rezultatele obținute se trec într-un tabel și se reprezintă grafic dependența curentului de saturație de intensitatea curentului de emisie.

Pentru măsurarea curentului de ionizare $I_c$ ca funcție de tensiunea tubului $I$ se parcurg pașii:

Pasul 1: Se setează curentul de emisie al tubului la $I=1 mA$.\\
Pasul 2: Se setează tensiunea condensatorului la $140V$.\\
Pasul 3: Se marește tensiunea $U$ în trepte de $2.5 kV$ începand de la $5kV$ pană la $35kV$.\\
Pasul 4: Rezultatele obținute se trec într-un tabel și se reprezintă grafic dependența curentului de saturație de tensiunea tubului.
}

{\section{Rezultate}
Pentru măsurarea curentului de ionizare ca funcție de curentul de emisie s-a ales tensiunea condensatorului de $140V$.
În urma masurătoriilor au rezultat tabelele \ref{tab1} \ref{tab2} și figuriile \ref{b} \ref{c}.
Urmărind figura \ref{b}, observăm că dependența intensitații curentului de ionizare de saturație $I_c$ față de curentul de emisie este una liniară. Adică pentru tensiunea maxima a tubului de emisie de raze X ($U=35kV$) intensitatea curentului de saturație este proportională cu intensitatea curentului de emise deci și a numărului de sarcini electrice create. Putem astfel trage concluzia că la tensiunea constantă de $U=35kV$ a sursei de radiații X numărul de sarcini electrice crește direct proporțional cu intensitatea curentului prin sursa de raze X.

Privind acum la \ref{b}, observăm că dependența intensității curentului de ionizare de saturație $I_c$ față de tensiunea tubului este una exponențială. Adică pentru intensitatea maximă a curentului prin tubul de emisie de raze X ($I=1mA$), intensitatea curentului crește exponențial cu tensiunea sursei de radiații X.
Putem astfel trage concluzia că la intensitatea constantă de $I=1mA$ a sursei de radiații X, numărul de sarcini electrice crește exponential cu tensiunea sursei de radiații X.



\begin{table}[H]
\begin{center}
\begin{tabular}{|c|c|c|c|} 
 \hline
 Nr. crt & I [mA] & U[V]  &$I_c$[nA]  \\ 
 \hline\hline
1 & 0 & 0 & 0  \\
 \hline
 2 & 0.1 & 0.61 &  0.61  \\
 \hline
 3 & 0.2 &  1.15 &  1.15  \\
 \hline
 4 & 0.3 & 1.57 &1.57 \\
 \hline
 5 & 0.4 & 1.96 & 1.96  \\
 \hline
 6 & 0.5 & 2.20 & 2.20  \\
 \hline
 7 & 0.6 & 2.64 & 2.64  \\
 \hline
 8 & 0.7 & 2.96 & 2.96  \\
 \hline
 9 & 0.8 & 3.20 & 3.20 \\
 \hline
 10 & 0.9 & 3.71 & 3.71  \\
 \hline
 11 & 1.0 & 4.06 & 4.06  \\
 \hline
\end{tabular}
\caption{\label{tab1}Curentul de ionizare de saturație $I_c$ ca funcție de curentul de emisie I.}
\end{center}
\end{table}


\begin{table}[H]
\begin{center}
\begin{tabular}{|c|c|c|c|} 
 \hline
 Nr. crt & U [kV] & U[V]  &$I_c$[nA]  \\ 
 \hline\hline
1 & 5 & 0.0015 &  0.0015  \\
 \hline
 2 & 7.5 & 0.0031 &0.0031  \\
 \hline
 3 & 10 &  0.0041 & 0.0041  \\
 \hline
 4 & 12.5 &0.015 & 0.015 \\
 \hline
 5 & 15& 0.13 & 0.13\\
 \hline
 6 & 17.5 & 0.37 &  0.37  \\
 \hline
 7 & 20 & 0.66 & 0.66  \\
 \hline
 8 & 22.5 & 1.01 & 1.01 \\
 \hline
 9 & 25 & 1.46 & 1.46 \\
 \hline
 10 & 27.5 & 2.10& 2.10  \\
 \hline
 10 & 30 & 2.65 &  2.65  \\
 \hline
 10 & 32.5 & 3.30 & 3.30  \\
 \hline
 10 & 35 & 3.97 & 3.97 \\
 \hline

\end{tabular}
\caption{\label{tab2}Curentul de ionizare de saturație $I_c$ ca funcție tensiunea tubului U.}
\end{center}
\end{table}



\begin{figure}[H]
\centering
  \includegraphics[width=0.9\textwidth]{b.png}
  \caption{\label{b}Curentul de ionizare de saturație $I_c$ ca funcție de curentul de emisie a sursei de radiații X.}
\end{figure}
\begin{figure}[H]
\centering
  \includegraphics[width=0.9\textwidth]{c.png}
  \caption{\label{c}Curentul de ionizare de saturație $I_c$ ca funcție tensiunea sursei de radiații X.}
\end{figure}

}







{\section{Concluzii}
În această lucrarea am prezentat o metodă de punere în evidență a generării de sarcini electrice în aer cu ajutorul radiaților X. Am arătat că numărului de sarcini electrice create depinde linear de intensitatea curentului de emisie a sursei de radiații X și exponențial de tensiunea sursei de radiații X.
}

\printbibliography

\end{document}