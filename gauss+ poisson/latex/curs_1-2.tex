\documentclass[12pt]{article}   %12 point font for Times New Roman


\usepackage{xcolor}
\usepackage{color}
\usepackage{float}
\usepackage{mathrsfs}
\usepackage{revsymb}
\usepackage{booktabs}
\usepackage{multirow}
\usepackage{rotating}
\usepackage{amssymb}
\usepackage{nicefrac}
\usepackage{graphicx}
\usepackage{epsfig}
\usepackage{mathrsfs}
\usepackage{xcolor}
\usepackage{amsmath}
\usepackage{amsthm}
\usepackage{graphicx}  %for images and plots
\usepackage[a4paper, left=2.5cm, right=2.5cm, top=2cm, bottom=2cm]{geometry}
\usepackage{setspace}  %use this package to set linespacing as desired
\usepackage{times}  %set Times New Roman as the font
\usepackage[explicit]{titlesec}  %title control and formatting

\usepackage[titles]{tocloft}  %table of contents control and formatting
\usepackage[backend=bibtex, sorting=none, bibstyle=ieee]{biblatex}  %reference manager
\usepackage[bookmarks=true, hidelinks]{hyperref}
\usepackage[page]{appendix}  %for appendices
\usepackage{rotating}  %for rotated, landscape images
\usepackage[normalem]{ulem}  %for italicized text
\usepackage{tabto}

%\newcommand{\sectionbreak}{\clearpage}

\usepackage[utf8]{inputenc}

\usepackage{amsmath}
\usepackage{verbatim}
\usepackage{hyperref}
\usepackage{url}
\usepackage{pdfcomment}
\usepackage{multirow}
\usepackage{gensymb}
\usepackage{array}

\usepackage[export]{adjustbox}

\usepackage[toc]{glossaries}

\makeglossaries

\usepackage{afterpage}

%\newcommand\blankpage{%
%	\null
%	\thispagestyle{empty}%
%	\addtocounter{page}{-1}%
%	\newpage}


%TABLE STUFF

%END TABLE STUFF

\DeclareUnicodeCharacter{2212}{-}

\DeclareUnicodeCharacter{223C}{~}

\makeatletter
\newcommand\footnoteref[1]{\protected@xdef\@thefnmark{\ref{#1}}\@footnotemark}
\makeatother

\makeatletter
\newcounter{subsubparagraph}[subparagraph]
\renewcommand\thesubsubparagraph{%
	\thesubparagraph.\@arabic\c@subsubparagraph}
\newcommand\subsubparagraph{%
	\@startsection{subsubparagraph}    % counter
	{6}                              % level
	{\parindent}                     % indent
	{3.25ex \@plus 1ex \@minus .2ex} % beforeskip
	{-1em}                           % afterskip
	{\normalfont\normalsize\bfseries}}
\newcommand\l@subsubparagraph{\@dottedtocline{6}{10em}{5em}}
\newcommand{\subsubparagraphmark}[1]{.}
\makeatother

\usepackage[english]{babel}

\newtheorem{theorem}{Addendum}
%\newlength{\overwritelength}
%\newlength{\minimumoverwritelength}
%\setlength{\minimumoverwritelength}{1cm}
%\newcommand{\overwrite}[3][red]{%
%	\settowidth{\overwritelength}{$#2$}%
%	\ifdim\overwritelength<\minimumoverwritelength%
%	\setlength{\overwritelength}{\minimumoverwritelength}\fi%
%	\stackrel
%	{%
%		\begin{minipage}{\overwritelength}%
%			\color{#1}\centering\small #3\\%
%			\rule{1pt}{9pt}%
%	\end{minipage}}
%	{\colorbox{#1!50}{\color{black}$\displaystyle#2$}}}

\newlength{\overwritelength}
\newlength{\minimumoverwritelength}
\setlength{\minimumoverwritelength}{1cm}
\newcommand{\overwrite}[3][red]{%
	\settowidth{\overwritelength}{$#2$}%
	\ifdim\overwritelength<\minimumoverwritelength%
	\setlength{\overwritelength}{\minimumoverwritelength}\fi%
	\stackrel
	{%
		\begin{minipage}{\overwritelength}%
			\color{#1}\centering\small #3\\%
			\rule{1pt}{9pt}%
	\end{minipage}}
	{\colorbox{#1!50}{\color{black}$\displaystyle#2$}}}

%%%%%%%%%%%%%%%%%%%%%%%%%%%%%%%%%%%
% Bibliography
%%%%%%%%%%%%%%%%%%%%%%%%%%%%%%%%%%%

%Add your bibliography file here
\bibliography{refc1}


% prevent certain fields in references from printing in bibliography
\AtEveryBibitem{\clearfield{issn}}
\AtEveryBibitem{\clearlist{issn}}

\AtEveryBibitem{\clearfield{language}}
\AtEveryBibitem{\clearlist{language}}

\AtEveryBibitem{\clearfield{doi}}
\AtEveryBibitem{\clearlist{doi}}

\AtEveryBibitem{\clearfield{url}}
\AtEveryBibitem{\clearlist{url}}

\AtEveryBibitem{%
	\ifentrytype{online}
	{}
	{\clearfield{urlyear}\clearfield{urlmonth}\clearfield{urlday}}}

%%%%%%%%%%%%%%%%%%%%%%%%%%%%%%%%%%
%paragraph indent
%%%%%%%%%%%%%%%%%%%%%%%%%%%%%%%%%%

\setlength{\parindent}{4em}
\setlength{\parskip}{1em}

%%%
%The subsubsub...subsection format
%%%


\title{Distribuții statistice: Poisson și Gauss}
\author{Ștefan-Răzvan~Anton\\ Anul 3, Grupa 1334,\\ Facultatea de Științe Aplicate}

\begin{document}

\maketitle



{\section{Scopul lucrării}
%
1.  Realizarea experimentală a unei analize statistice.\\
2.  Determinarea unor mărimi caracteristice unei distribuții statistice.\\
3.  Asemănări și deosebiri între distribuțiile Poisson și Gauss.
}
%
{\section{Modul de lucru}
În această lucrare, dorim să caracterizăm distribuția statistică a pulsurilor electrice obținute pe un detector cu scintilație expus radiațiilor emise de o probă ce conține ${}^{241} Am$, atunci când proba se afla la distanțe diferite fața de detector.
Astfel, vom face două masuratori.
\subsection{Masuratoarea 1}
{
Pentru aceata masuratoare este necesar să inregistrăm $1024$ de măsuratori pentru un timp de numărare de $1s$.
Proba de ${}^{241} Am$ se va poziționa la o distanță apoximativă de $2cm$ față de detector, iar numărul mediu de impulsuri pentru un timp de numărare de $1s$ sa fie $3-5$.
}
\subsection{Masuratoarea 2}
{
Pentru aceata masuratoare este necesar să inregistrăm $2048$ de măsuratori pentru un timp de numărare de $1s$.
Proba de ${}^{241} Am$ se va poziționa la o distanță apoximativă de $1.5cm$ față de detector, iar numărul mediu de impulsuri pentru un timp de numărare de $1s$ sa fie $12-18$.
}
Dupa realizarea celor două masuratori se vor reprezenta histogramele masuratoriilor si se vor fita catat cu o distributie normala scalata cat si cu o distributie Poisson scalata. Curbele rezultate se vor reprezenta pe acelasi grafic cu histograma pe care au fost fitate. \\
In continuare, plecand de la datele masuratoriilor se vor calcula si interpreta marimiile: valoare medie, dispersie, abatere standard, asimetria, si aplatizarea
}



{\section{Rezultate}
In urma celor doua masuratori au rezultat datele reprezentate grafic in Fig \ref{fig1} si Fig \ref{fig2}
\begin{figure}[H]
\centering
  \includegraphics[width=0.9\textwidth]{poiss_mas}
  \caption{\label{fig1}Rezultatul masuratorii 1 (Evenimente pe secunda).}
\end{figure}

\begin{figure}[H]
\centering
  \includegraphics[width=0.9\textwidth]{gauss_mas}
  \caption{\label{fig2}Rezultatul masuratorii 2 (Evenimente pe secunda).}
\end{figure}



Prin fitare cu o distributie normala scalata obtinem 

\begin{eqnarray}\nonumber
f_1 &=& 1036 N(3.16,1.87^2)\,,\\ \nonumber
f_2 &=& 2045 N(13.5,3.73^2)\,,
\end{eqnarray}
unde $N(a,b)$ este distrubutia normala cu valaorea medie $a$ si deviatia standard $b$.\\


Prin fitare cu o distributie Poisson obtinem 

\begin{eqnarray}\nonumber
f_1 &=& 1023 P(3.43)\,,\\ \nonumber
f_2 &=& 2039 P(13.8)\,,
\end{eqnarray}
unde $P(a)$ este distrubutia Poisson cu valaorea medie $a$.


Curbele obtinute in urma fitariilor impreuna cu histogramele pe care au fost fitate au fost reprezentate grafic in Fig \ref{fig3} si Fig \ref{fig4}


\begin{figure}[H]
\centering
  \includegraphics[width=0.9\textwidth]{poiss_hist}
  \caption{\label{fig3}Histograma + Norm + Poiss Masuratoarea 1.}
\end{figure}

\begin{figure}[H]
\centering
  \includegraphics[width=0.9\textwidth]{gauss_hist}
  \caption{\label{fig4}Histograma + Norm + Poiss Masuratoarea 2.}
\end{figure}






\begin{table}[H]
\begin{center}
\begin{tabular}{|c|c|c|c|c|c|c|} 
 \hline
 Masuratoare & val. medie & dispersie & deviatie standard & asimetria & aplatizarea& interval \\ 
 \hline\hline
 1 & 13.84 & 14.01 & 3.74 & 0.29 & -0.02 & 4-28 \\
 \hline
 2 & 3.41 & 3.44 & 1.85 & 0.54 & 0.34 & 0-11 \\

 \hline
\end{tabular}
\caption{\label{tab1}Unele marimi caracteristice pentru distributia statsitica a masuratoriilor.}
\end{center}
\end{table}
}







{\section{Concluzii}

}
\end{document}