\documentclass[12pt]{article}   %12 point font for Times New Roman


\usepackage{xcolor}
\usepackage{color}
\usepackage{float}
\usepackage{mathrsfs}
\usepackage{revsymb}
\usepackage{booktabs}
\usepackage{multirow}
\usepackage{rotating}
\usepackage{amssymb}
\usepackage{nicefrac}
\usepackage{graphicx}
\usepackage{epsfig}
\usepackage{mathrsfs}
\usepackage{xcolor}
\usepackage{amsmath}
\usepackage{amsthm}
\usepackage{graphicx}  %for images and plots
\usepackage[a4paper, left=2.5cm, right=2.5cm, top=2cm, bottom=2cm]{geometry}
\usepackage{setspace}  %use this package to set linespacing as desired
\usepackage{times}  %set Times New Roman as the font
\usepackage[explicit]{titlesec}  %title control and formatting

\usepackage[titles]{tocloft}  %table of contents control and formatting
\usepackage[backend=bibtex, sorting=none, bibstyle=ieee]{biblatex}  %reference manager
\usepackage[bookmarks=true, hidelinks]{hyperref}
\usepackage[page]{appendix}  %for appendices
\usepackage{rotating}  %for rotated, landscape images
\usepackage[normalem]{ulem}  %for italicized text
\usepackage{tabto}

%\newcommand{\sectionbreak}{\clearpage}

\usepackage[utf8]{inputenc}

\usepackage{amsmath}
\usepackage{verbatim}
\usepackage{hyperref}
\usepackage{url}
\usepackage{pdfcomment}
\usepackage{multirow}
\usepackage{gensymb}
\usepackage{array}

\usepackage[export]{adjustbox}

\usepackage[toc]{glossaries}

\makeglossaries

\usepackage{afterpage}

%\newcommand\blankpage{%
%	\null
%	\thispagestyle{empty}%
%	\addtocounter{page}{-1}%
%	\newpage}


%TABLE STUFF

%END TABLE STUFF

\DeclareUnicodeCharacter{2212}{-}

\DeclareUnicodeCharacter{223C}{~}

\makeatletter
\newcommand\footnoteref[1]{\protected@xdef\@thefnmark{\ref{#1}}\@footnotemark}
\makeatother

\makeatletter
\newcounter{subsubparagraph}[subparagraph]
\renewcommand\thesubsubparagraph{%
	\thesubparagraph.\@arabic\c@subsubparagraph}
\newcommand\subsubparagraph{%
	\@startsection{subsubparagraph}    % counter
	{6}                              % level
	{\parindent}                     % indent
	{3.25ex \@plus 1ex \@minus .2ex} % beforeskip
	{-1em}                           % afterskip
	{\normalfont\normalsize\bfseries}}
\newcommand\l@subsubparagraph{\@dottedtocline{6}{10em}{5em}}
\newcommand{\subsubparagraphmark}[1]{.}
\makeatother

\usepackage[english]{babel}

\newtheorem{theorem}{Addendum}
%\newlength{\overwritelength}
%\newlength{\minimumoverwritelength}
%\setlength{\minimumoverwritelength}{1cm}
%\newcommand{\overwrite}[3][red]{%
%	\settowidth{\overwritelength}{$#2$}%
%	\ifdim\overwritelength<\minimumoverwritelength%
%	\setlength{\overwritelength}{\minimumoverwritelength}\fi%
%	\stackrel
%	{%
%		\begin{minipage}{\overwritelength}%
%			\color{#1}\centering\small #3\\%
%			\rule{1pt}{9pt}%
%	\end{minipage}}
%	{\colorbox{#1!50}{\color{black}$\displaystyle#2$}}}

\newlength{\overwritelength}
\newlength{\minimumoverwritelength}
\setlength{\minimumoverwritelength}{1cm}
\newcommand{\overwrite}[3][red]{%
	\settowidth{\overwritelength}{$#2$}%
	\ifdim\overwritelength<\minimumoverwritelength%
	\setlength{\overwritelength}{\minimumoverwritelength}\fi%
	\stackrel
	{%
		\begin{minipage}{\overwritelength}%
			\color{#1}\centering\small #3\\%
			\rule{1pt}{9pt}%
	\end{minipage}}
	{\colorbox{#1!50}{\color{black}$\displaystyle#2$}}}

%%%%%%%%%%%%%%%%%%%%%%%%%%%%%%%%%%%
% Bibliography
%%%%%%%%%%%%%%%%%%%%%%%%%%%%%%%%%%%

%Add your bibliography file here
\bibliography{refc1}


% prevent certain fields in references from printing in bibliography
\AtEveryBibitem{\clearfield{issn}}
\AtEveryBibitem{\clearlist{issn}}

\AtEveryBibitem{\clearfield{language}}
\AtEveryBibitem{\clearlist{language}}

\AtEveryBibitem{\clearfield{doi}}
\AtEveryBibitem{\clearlist{doi}}

\AtEveryBibitem{\clearfield{url}}
\AtEveryBibitem{\clearlist{url}}

\AtEveryBibitem{%
	\ifentrytype{online}
	{}
	{\clearfield{urlyear}\clearfield{urlmonth}\clearfield{urlday}}}

%%%%%%%%%%%%%%%%%%%%%%%%%%%%%%%%%%
%paragraph indent
%%%%%%%%%%%%%%%%%%%%%%%%%%%%%%%%%%

\setlength{\parindent}{4em}
\setlength{\parskip}{1em}

%%%
%The subsubsub...subsection format
%%%


\title{Analiza spectroscopică a unor lămpi cu descărcare}
\author{Ștefan-Răzvan~Anton\\ Anul 3, Grupa 1334,\\ Facultatea de Științe Aplicate}

\begin{document}

\maketitle



{\section{Scopul lucrării}
%
1.  Înțelegerea structurii și funcționării unei lămpi cu descărcare în gaze.\\
2.  Utilizarea unei metode spectroscopice pentru caracterizarea și analiza unor plasme de desărcare.\\
3.  Identificarea elementelor chimice din plasma de descărcare.
}
%
{\section{Principiul fizic}

}
{\section{Montajul experimental}



\begin{figure}[H]
\centering
  \includegraphics[width=0.9\textwidth]{sistem}
  \caption{\label{figs}Montajul experimental.}
\end{figure}
}


{\section{Modul de lucru}

}

{\section{Rezultate}


\begin{table}[H]
\begin{center}
\begin{tabular}{|c|c|c|c|c|} 
 \hline
 Culoare & Unghi (grade) & $\lambda$ teoretică(nm)  & $\lambda$ experimentală (nm) & Tranziție  \\ 
 \hline\hline
 roșu & 123.5 & 668 & 639 &  ${3}^{1}D_2 \rightarrow {2}^{1}P_1$ \\
 \hline
 portocaliu & 124.3 & 587 & 593 &  ${3}^{3}D \rightarrow {2}^{3}P$  \\
 \hline
 verde & 125.85 &  504 & 503 &  ${4}^{1}S_0 \rightarrow {2}^{1}P_1$  \\
 \hline
 verde & 126.35 &492 & 486 & ${4}^{1}D_2 \rightarrow {2}^{1}P_1$  \\
 \hline
 albastru & 127.1 & 447 & 431 & ${4}^{3}D \rightarrow {2}^{3}P$  \\
 \hline
 violat & 127.25 & 438 & 422 & ${5}^{1}D_2 \rightarrow {2}^{1}P_1$  \\
 \hline
\end{tabular}
\caption{\label{tab1}Tabelul pentru etalonare cu ajutorul lampii cu descărcare în Heliu.}
\end{center}
\end{table}



\begin{figure}[H]
\centering
  \includegraphics[width=0.9\textwidth]{regr}
  \caption{\label{figs}Dreapta de regresie și valorile experimentale pentru lampa cu descărcare în Heliu.}
\end{figure}

\begin{table}[H]
\begin{center}
\begin{tabular}{|c|c|c|c|c|} 
 \hline
 Culoare & Unghi (grade) & $\lambda$ teoretică(nm)  & $\lambda$ experimentală (nm) & Tranziție  \\ 
 \hline\hline
 roșu & 124 & 643 & 620 &  ${5}^{1}D_2 \rightarrow {5}^{1}P_1$ \\
 \hline
 verde & 125.6 & 508 & 529 &  ${6}^{3}S_1 \rightarrow {5}^{3}P_2$  \\
 \hline
 albastru & 126.25 & 479 & 492 &  ${6}^{3}S_1 \rightarrow {5}^{3}P_1$  \\
 \hline
 ablastru & 126.65 & 467 & 469 &  ${6}^{3}S_1 \rightarrow {5}^{3}P_0$ \\
 \hline
 mov & 127.2 & 441 & 438 & ${6}^{1}S_0 \rightarrow {5}^{3}P_1$  \\
 \hline
\end{tabular}
\caption{\label{tab2}Determinarea valorii lungimii de undă penntru lampa cu descărcare in Cadmiu.}
\end{center}
\end{table}


\begin{table}[H]
\begin{center}
\begin{tabular}{|c|c|c|c|c|} 
 \hline
 Culoare & Unghi (grade) & $\lambda$ teoretică(nm)  & $\lambda$ experimentală (nm) & Tranziție  \\ 
 \hline\hline
 roșu & 123.85 & 636 & 629 &  ${4}^{1}D_2 \rightarrow {4}^{1}P_1$ \\
 \hline
 verde & 125.5 & 518 & 535 &  ${6}^{1}S_0 \rightarrow {4}^{1}P_1$  \\
 \hline
 albastru & 126.05 & 481 & 504 &  ${5}^{3}S_1 \rightarrow {4}^{3}P_2$  \\
 \hline
 ablastru & 126.55 & 472 & 475 &  ${5}^{3}S_1 \rightarrow {4}^{3}P_1$ \\
 \hline
 albastru & 126.8 & 468 & 461 & ${5}^{3}S_1 \rightarrow {4}^{3}P_0$  \\
 \hline
\end{tabular}
\caption{\label{tab3}Determinarea valorii lungimii de undă penntru lampa cu descărcare in Zinc..}
\end{center}
\end{table}





}






{\section{Concluzii}
}
\end{document}